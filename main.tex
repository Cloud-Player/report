% Dieses Dokument muss mit PDFLatex gesetzt werden
% Vorteil: Grafiken koennen als jpg, png, ... verwendet werden
%          und die Links im Dokument sind auch gleich richtig
%
%Ermöglicht \\ bei der Titelseite (z.B. bei supervisor)
%Siehe https://github.com/latextemplates/uni-stuttgart-cs-cover/issues/4
\RequirePackage{kvoptions-patch}

%English:
\let\ifdeutsch\iffalse
\let\ifenglisch\iftrue

%German:
%\let\ifdeutsch\iftrue
%\let\ifenglisch\iffalse

%
\ifenglisch
	\PassOptionsToClass{numbers=noenddot}{scrbook}
\else
	%()Aus scrguide.pdf - der Dokumentation von KOMA-Script)
	%Nach DUDEN steht in Gliederungen, in denen ausschließlich arabische Ziffern für die Nummerierung
	%verwendet werden, am Ende der Gliederungsnummern kein abschließender Punkt
	%(siehe [DUD96, R3]). Wird hingegen innerhalb der Gliederung auch mit römischen Zahlen
	%oder Groß- oder Kleinbuchstaben gearbeitet, so steht am Ende aller Gliederungsnummern ein
	%abschließender Punkt (siehe [DUD96, R4])
	\PassOptionsToClass{numbers=autoendperiod}{scrbook}
\fi

%Warns about outdated packages and missing caption delcarations
%See https://www.ctan.org/pkg/nag
\RequirePackage[l2tabu, orthodox]{nag}

%Neue deutsche Trennmuster
%Siehe http://www.ctan.org/pkg/dehyph-exptl und http://projekte.dante.de/Trennmuster/WebHome
%Nur für pdflatex, nicht für lualatex
\RequirePackage{ifluatex}
\ifluatex
%do not load anything
\else
	\ifdeutsch
		\RequirePackage[ngerman=ngerman-x-latest]{hyphsubst}
	\fi
\fi

\documentclass[
               fontsize=12pt, %Default: 11pt, bei Linux Libertine zu klein zum Lesen
% BEGINN: Optionen für typearea
               % paper=a4, CUSTOM OFF
               oneside,  % fuer die Betrachtung am Schirm ungeschickt
               BCOR=3mm, % Bindekorrektur
               DIV=12,   % je höher der DIV-Wert, desto mehr geht auf eine Seite. Gute werde sind zwischen DIV=12 und DIV=15
               % headinclude=true, CUSTOM OFF
               % footinclude=false, CUSTOM OFF
% ENDE: Optionen für typearea
%               titlepage,
               bibliography=totoc,
%               idxtotoc,   %Index ins Inhaltsverzeichnis
%                liststotoc, %List of X ins Inhaltsverzeichnis, mit liststotocnumbered werden die Abbildungsverzeichnisse nummeriert
               headsepline,
               cleardoublepage=empty,
               parskip=half,
%               draft    % um zu sehen, wo noch nachgebessert werden muss - wichtig, da Bindungskorrektur mit drin
               final   % ACHTUNG! - in pagestyle.tex noch Seitenstil anpassen
               ]{scrbook}


\input{preambel/options}
\usepackage{titlesec}
\titleformat{\paragraph}
{\normalfont\normalsize}{\theparagraph}{1em}{}
\titlespacing*{\paragraph}
{0pt}{1.5ex plus 1ex minus .2ex}{0pt}

% quotes -> \say{}
\usepackage{dirtytalk}

\usepackage[utf8]{inputenc}

\usepackage{multicol}
\usepackage{courier}
\usepackage{listings} %For code in appendix
\lstset
{ %Formatting for code in appendix
    basicstyle=\footnotesize\ttfamily,
    numbers=left,
    stepnumber=1,
    showstringspaces=false,
    tabsize=1,
    breaklines=true,
    breakatwhitespace=false,
    xleftmargin=2em,
    escapechar=|
}

% tables
\usepackage{tabu}

%Der untere Rand darf "flattern"
\raggedbottom

%%%
% Wie tief wird das Inhaltsverzeichnis aufgeschlüsselt
% 0 --\chapter
% 1 --\section % fuer kuerzeres Inhaltsverzeichnis verwenden - oder minitoc benutzen
% 2 --\subsection
% 3 --\subsubsection
% 4 --\paragraph
\setcounter{tocdepth}{1}
%
%%%

\makeindex

%Angaben in die PDF-Infos uebernehmen
\makeatletter
\hypersetup{
            pdftitle={Developing a Distributed Media Streaming Application},
            pdfauthor={Alexander Zarges, Nicolas Drebenstedt},
            pdfkeywords={},
            pdfsubject={}
}
\makeatother

\newacronym[plural={Application Programming Interfaces},shortplural={APIs}]{api}{API}{Application Programming Interface}
\newacronym{alm}{ALM}{Application Layer Multicast}
\newacronym{cbr}{CBR}{Constant Bitrate}
\newacronym[plural={Content Delivery Networks},shortplural={CDNs}]{cdn}{CDN}{Content Delivery Network}
\newacronym{cernet}{CERNET}{China Education and Research Network}
\newacronym{ccn}{CCN}{Content–Centric Networking}
\newacronym{dag}{DAG}{Directed Acyclic Graph}
\newacronym{dash}{DASH}{Dynamic Adaptive Streaming over HTTP}
\newacronym{dhcp}{DHCP}{Dynamic Host Configuration Protocol}
\newacronym{dht}{DHT}{Distributed Hash Table}
\newacronym{dns}{DNS}{Domain Name System}
\newacronym{drm}{DRM}{Digital Rights Management}
\newacronym{dtls}{DTLS}{Datagram Transport Layer Security}
\newacronym{eq}{EQ}{Echo Link Quality}
\newacronym{gvbr}{GVBR}{Greedy Variable Bitrate}
\newacronym{hlpsp}{HLPSP}{Hybrid Live P2P Streaming Protocol}
\newacronym{hls}{HLS}{HTTP Live Streaming}
\newacronym{html}{HTML}{Hypertext Markup Language}
\newacronym{http}{HTTP}{Hypertext Transfer Protocol}
\newacronym{icn}{ICN}{Information–Centric Networking}
\newacronym{ice}{ICE}{Interactive Connectivity Establishment}
\newacronym{ietf}{IETF}{Internet Engineering Task Force}
\newacronym{ipfs}{IPFS}{InterPlanetary File System}
\newacronym{ip}{IP}{Internet Protocol}
\newacronym{isp}{ISP}{Internet Service Provider}
\newacronym{ixp}{IXP}{Internet Exchange Point}
\newacronym{isdn}{ISDN}{Integrated Services Digital Network}
\newacronym{isup}{ISUP}{\gls{isdn} User Part}
\newacronym{js}{JS}{JavaScript}
\newacronym{mpegdash}{MPEG–DASH}{Dynamic Adaptive Streaming over HTTP}
\newacronym[plural={Multi Point Relays},shortplural={MPRs}]{mpr}{MPR}{Multi Point Relay}
\newacronym[plural={Mobile Network Operators},shortplural={MNOs}]{mno}{MNO}{Mobile Network Operator}
\newacronym{mse}{MSE}{Media Source Extensions}
\newacronym{nat}{NAT}{Network Address Translation}
\newacronym{ndn}{NDN}{Named Data Networking}
\newacronym{olsr}{OLSR}{Optimized Link State Routing}
\newacronym[plural={Originator Messages},shortplural={OGMs}]{ogm}{OGM}{Originator Message}
\newacronym{osi}{OSI}{Open Systems Interconnection}
\newacronym{os}{OS}{Operating System}
\newacronym{p2p}{P2P}{Peer–to–Peer}
\newacronym{qos}{QoS}{Quality of Service}
\newacronym{qoe}{QoE}{Quality of Experience}
\newacronym{rq}{RQ}{Receive Link Quality}
\newacronym{rtcp}{RTCP}{Real–Time Control Protocol}
\newacronym{rtp}{RTP}{Real–Time Transport Protocol}
\newacronym{rtmp}{RTMP}{Real–Time Messaging Protocol}
\newacronym[plural={Software Development Kits},shortplural={SDKs}]{sdk}{SDK}{Software Development Kit}
\newacronym{smtp}{SMTP}{Simple Mail Transfer Protocol}
\newacronym{srtcp}{SRTCP}{Secure Real–time Control Transport Protocol}
\newacronym{srtp}{SRTP}{Secure Real–time Transport Protocol}
\newacronym{sdp}{SDP}{Session Description Protocol}
\newacronym{sctp}{SCTP}{Stream Control Transport Protocol}
\newacronym{sip}{SIP}{Session Initiation Protocol}
\newacronym{stun}{STUN}{Session Traversal Utilities for NAT}
\newacronym{tcp}{TCP}{Transmission Control Protocol}
\newacronym{tc}{TC}{Topology Control}
\newacronym{tq}{TQ}{Transmit Link Quality}
\newacronym{ttl}{TTL}{Time To Live}
\newacronym{turn}{TURN}{Traversal Using Relays around NAT}
\newacronym{udp}{UDP}{User Datagram Protocol}
\newacronym{url}{URL}{Uniform Resource Identifier}
\newacronym{ui}{UI}{User Interface}
\newacronym{vbr}{VBR}{Variable Bitrate}
\newacronym{vod}{VOD}{Video–on–Demand}
\newacronym{vpn}{VPN}{Virtual Private Network}
\newacronym{voip}{VoIP}{Voice over Internet Protocol}
\newacronym{w3c}{W3C}{World Wide Web Consortium}
\newacronym[plural={WebSockets}]{ws}{WS}{WebSocket}
\newacronym{webrtc}{WebRTC}{Web Real–Time Communication}
\newacronym{xor}{XOR}{Exclusive Or}


\begin{document}

%tex4ht-Konvertierung verschönern
\iftex4ht
% tell tex4ht to create picures also for formulas starting with '$'
% WARNING: a tex4ht run now takes forever!
\Configure{$}{\PicMath}{\EndPicMath}{} 
%$ % <- syntax highlighting fix for emacs
\Css{body {text-align:justify;}}

%conversion of .pdf to .png
\Configure{graphics*}  
         {pdf}  
         {\Needs{"convert \csname Gin@base\endcsname.pdf  
                               \csname Gin@base\endcsname.png"}%  
          \Picture[pict]{\csname Gin@base\endcsname.png}%  
         }  
\fi

%Tipp von http://goemonx.blogspot.de/2012/01/pdflatex-ligaturen-und-copynpaste.html
%siehe auch http://tex.stackexchange.com/questions/4397/make-ligatures-in-linux-libertine-copyable-and-searchable
%
%ONLY WORKS ON MiKTeX
%On other systems, download glyphtounicode.tex from http://pdftex.sarovar.org/misc/
%
\input glyphtounicode.tex
\pdfgentounicode=1

%\VerbatimFootnotes %verbatim text in Fußnoten erlauben. Geht normalerweise nicht.

%wird fuer Tabellen benötigt (z.B. >{centering\RBS}p{2.5cm} erzeugt einen zentrierten 2,5cm breiten Absatz in einer Tabelle
\newcommand{\RBS}{\let\\=\tabularnewline}

%To avoid issues with Springer's \mathplus
%See also http://tex.stackexchange.com/q/212644/9075
\providecommand\mathplus{+}

%% typoraphisch richtige Abkürzungen
\newcommand{\zB}[0]{z.\,B.\xspace}
\newcommand{\bzw}[0]{bzw.\xspace}
\newcommand{\usw}[0]{usw.\xspace}
\renewcommand{\dh}[0]{d.\,h.\xspace}

%from hmks makros.tex - \indexify
\newcommand{\toindex}[1]{\index{#1}#1}
%
\newcommand{\dotcup}{\ensuremath{\,\mathaccent\cdot\cup\,}} %Tipp aus The Comprehensive LaTeX Symbol List
%
%Anstatt $|x|$ $\abs{x}$ verwenden. Die Betragsstriche skalieren automatisch, falls "x" etwas größer sein sollte...
\newcommand{\abs}[1]{\left\lvert#1\right\rvert}
%
%für Zitate
\newcommand{\citeS}[2]{\cite[S.~#1]{#2}}
\newcommand{\citeSf}[2]{\cite[S.~#1\,f.]{#2}}
\newcommand{\citeSff}[2]{\cite[S.~#1\,ff.]{#2}}
\newcommand{\vgl}{vgl.\ }
\newcommand{\Vgl}{Vgl.\ }
%
\newcommand{\commentchar}{\ensuremath{/\mkern-4mu/}}
\algrenewcommand{\algorithmiccomment}[1]{\hfill $\commentchar$ #1}

% Seitengrößen - Gegen Schusterjungen und Hurenkinder...
\newcommand{\largepage}{\enlargethispage{\baselineskip}}
\newcommand{\shortpage}{\enlargethispage{-\baselineskip}}

% CUSTOM
\newcommand{\itembf}[1]{\item\textbf{#1}\hspace{3mm}}
\newcommand*{\rom}[1]{\uppercase\expandafter{\romannumeral #1\relax}}

\newcommand{\alice}{\textit{Alice}\xspace}
\newcommand{\bob}{\textit{Bob}\xspace}
\newcommand{\claire}{\textit{Claire}\xspace}
\newcommand{\don}{\textit{Don}\xspace}
\newcommand{\eve}{\textit{Eve}\xspace}
\newcommand{\zoe}{\textit{Zoe}\xspace}
\newcommand{\signal}{\textit{Signal}\xspace}
\newcommand{\newbie}{\textit{Newbie}\xspace}
\newcommand{\router}{\textit{Router}\xspace}
\newcommand{\peer}{\textit{Peer}\xspace}

\newcommand{\maxConnections}{$C_{max}$\xspace}
\newcommand{\minConnectionGoal}{$C_{goal_{min}}$\xspace}
\newcommand{\maxConnectionGoal}{$C_{goal_{max}}$\xspace}
\newcommand{\outStreamConnections}{$S_{out}$\xspace}
\newcommand{\inStreamConnections}{$S_{in}$\xspace}

\pagenumbering{arabic}
\Titelblatt

%Eigener Seitenstil fuer die Kurzfassung und das Inhaltsverzeichnis
\deftriplepagestyle{preamble}{}{}{}{}{}{\pagemark}
%Doku zu deftripstyle: scrguide.pdf
\pagestyle{preamble}
\renewcommand*{\chapterpagestyle}{preamble}

\makeatletter
\newenvironment{chapquote}[2][2em]
  {\setlength{\@tempdima}{#1}%
   \def\chapquote@author{#2}%
   \parshape 1 \@tempdima \dimexpr\textwidth-2\@tempdima\relax%
   \itshape}
  {\par\normalfont\hfill--\ \chapquote@author\hspace*{\@tempdima}\par\bigskip}
\makeatother


\pagebreak
\hspace{0pt}
\vfill

\begin{chapquote}{\textit{Wikipedia}}
`In cell biology, mitosis \textipa{[maI"toUsis]} is a part of the cell cycle when replicated chromosomes are separated into two new nuclei'
\end{chapquote}

\vfill
\hspace{0pt}
\pagebreak
\newpage

%Kurzfassung / abstract
%auch im Stil vom Inhaltsverzeichnis
\ifdeutsch
\section*{Kurzfassung}
\else
\section*{Abstract}
\fi
\ldots ... Short summary of the thesis ...
\cleardoublepage


% BEGIN: Verzeichnisse

\iftex4ht
\else
\microtypesetup{protrusion=false}
\fi

%%%
% Literaturverzeichnis ins TOC mit aufnehmen, aber nur wenn nichts anderes mehr hilft!
% \addcontentsline{toc}{chapter}{Literaturverzeichnis}
%
% oder zB
%\addcontentsline{toc}{section}{Abkürzungsverzeichnis}
%
%%%

%Produce table of contents
%
%In case you have trouble with headings reaching into the page numbers, enable the following three lines.
%Hint by http://golatex.de/inhaltsverzeichnis-schreibt-ueber-rand-t3106.html
%
%\makeatletter
%\renewcommand{\@pnumwidth}{2em}
%\makeatother
%
\tableofcontents

% Bei einem ungünstigen Seitenumbruch im Inhaltsverzeichnis, kann dieser mit
% \addtocontents{toc}{\protect\newpage}
% an der passenden Stelle im Fließtext erzwungen werden.

\listoffigures
%\listoftables

%Wird nur bei Verwendung von der lstlisting-Umgebung mit dem "caption"-Parameter benoetigt
%\lstlistoflistings 
%ansonsten:
\ifdeutsch
%\listof{Listing}{Verzeichnis der Listings}
\else
\listof{Listing}{List of Listings}
\fi

%mittels \newfloat wurde die Algorithmus-Gleitumgebung definiert.
%Mit folgendem Befehl werden alle floats dieses Typs ausgegeben
\ifdeutsch
%\listof{Algorithmus}{Verzeichnis der Algorithmen}
\else
%\listof{Algorithmus}{List of Algorithms}
\fi
%\listofalgorithms %Ist nur für Algorithmen, die mittels \begin{algorithm} umschlossen werden, nötig

% Abkürzungsverzeichnis
% \printnoidxglossaries CUSTOM
\printglossary[type=\acronymtype]
\printglossary

\iftex4ht
\else
%Optischen Randausgleich und Grauwertkorrektur wieder aktivieren
\microtypesetup{protrusion=true}
\fi

% END: Verzeichnisse


\renewcommand*{\chapterpagestyle}{scrplain}
\pagestyle{scrheadings}
\input{preambel/pagestyle}

%
%
% ** Hier wird der Text eingebunden **
%
\chapter{Introduction}
\section{Introduction}

A new kind of decentralised consumer applications has emerged out of the problems cloud services are facing today: network congestion,
server costs and privacy concerns. \textit{dApps} address those issues by replacing centralised server architectures with a decentralised, self-organising network of clients. The three core areas of cloud computing, communication, storage and computation, are mapped into the decentralised paradigm. Clients are connected via peer-to-peer networking and do distributed content-addressed routing. Storage of media and application code is provided by decentralised file systems. Computation of business logic and sensitive transactions are spread across multiple clients that build a chain of trust. Technologies like IPFS \footnote{IPFS} (distributed file system), Chord \cite{chord} (distributed routing) and Ethereum \footnote{Ethereum} (distributed computation) enable new ways to build the web application of the future – a web application that implements privacy by design and has no single point of failure.

Consumer clients have become more capable thanks to advances in mobile hardware and the maturity of browser engines and their integration with OS-native functionality. This allows developers access to device sensors, GPU programming and opens up more possibilities with the network stack. Modern browsers expose interfaces for real-time communication between server and client, called WebSockets \cite{rfc6455}, and between clients, called WebRTC \cite{webrtc-w3c}. The latter enables connections to other clients without routing the data packages through a central server. Clients can therefore establish a connection to another client by only using the available network infrastructure.

The proposed thesis aims to apply the decentralised paradigm to an existing music player web application \footnote{cloudPlayer} that currently can play music from external providers like YouTube. The music player is to be extended by a peer-to-peer radio broadcasting feature. Any peer can act as a \textit{broadcaster} and control the media queue and playback of its \textit{receiving} peers. In addition, the \textit{broadcaster} can broadcast live audio content for vocal announcements and creative input. The radio feature will rely on decentralised networking to broadcast the content.

Two main challenges arise from this endeavour – optimising the peer-to-peer network for multi-cast streaming and securing the integrity and privacy of broadcasting streams.

\section{Random Intro quotes}
\say{It can be expected that the number and importance of such distributed applications will grow with time as the Internet is penetrating more and more into our lives.} (https://www.cs.vu.nl/pub/steen/papers/wip-newscast.pdf \S1)

\say{The popularity of peer-to-peer systems in the last couple of years illustrates how the Internet is gradually shifting toward a distributed system that supports more than only client-server applications.}(https://link.springer.com/content/pdf/10.1007\%2F978-3-540-25840-7\_6.pdf \S1)

\chapter{Motivation}
- Kostenfaktor
- Zensur bla
- Network Congestion / Alle leute machen Videostreaming. Mit multicast kannste dir so einige petabytes sparen. Pakete müssen nicht über nicht immer über die ganze welt hüpfen, bleibt in der regio woe es produziert/konsumiert wird. (Facebook paper referenzieren, Youtube/Netflix produzieren peta byte traffic)


%\chapter{Background}
\section{Browser Media}

A true distributed media streaming application must also rely on distributed content generation. Users of the system must be able to record media on their connected devices and provide a stream to the network. Other users must be able to consume those streams of audio and video data with their respective devices. Considering the constraint, that the application should be executable in a browser, used media formats have to meet further requirements.

\subsection{Media Pipeline}
\label{browser-api}

Modern browsers implement a set of JavaScript \glspl{api} that comply with the {\textit{Media Capture and Streams}} draft \citep{media-capture-and-streams} of the \gls{w3c}. The draft defines an interface for media streams that can contain multiple tracks, which in turn can contain multiple channels. In its most common configuration a stream would consist of a video and audio track and the latter would contain two channels for a stereo signal. A media stream instance can either be created locally or originate from a network source. Either of those streams can be presented to the local user or streamed to a network connection. \citet{media-capture-and-streams} denominates these two usages as sources and sinks respectively.

\subsubsection{Stream Sources}

Streams can only be recorded locally after the application has acquired user consent from the browser and \gls{os}. The user media \gls{api} \cite{TODO} allows to constrain recording parameters, such as video resolution, frame rate or audio volume (see \ref{lst:recording-constraints}). The satisfiability of these constraints however, depends on browser, \gls{os} and device compatibility.

\begin{Listing}
\begin{lstlisting}
{
  "audio": {
    "autoGainControl": false,
    "channelCount": 2,
    "echoCancellation": true,
    "volume": {
      "ideal": 0.75
    }
  },
  "video": {
    "facingMode": {
      "exact": "user"
    },
    "frameRate": {
      "ideal": 60
    }
  }
}
\end{lstlisting}
\caption{Example media stream recording constraints for the getUserMedia API}
\label{lst:recording-constraints}
\end{Listing}

A media stream can also originate from a network source. \Gls{http} streaming and \gls{webrtc} connections are the two common approaches. The former shall be further discussed in section~\ref{subsec:http-streaming}, the latter in section~\ref{subsec:webrtc-media}.

\subsubsection{Stream Sinks}

The browser media pipeline provides that a stream ends with a so called sink. To display it to the local user, the \gls{html} standard \cite[\S4.7]{html-w3c} defines media tags such as \lstinline|<audio>| and \lstinline|<video>|. These tags feature a JavaScript \gls{api} for themselves to control playback and react to stream changes. A stream can be directly attached to the elements and the browser handles low—level playback features such as pre—loading, buffering and random access. To forward the stream to a remote system, a media stream can also be directed to an outgoing \gls{webrtc} connection. The \gls{api} of which states the browser should handle stream constrain negotiation like letting sender and receiver agree on a video resolution \cite[\S5.1]{webrtc-w3c}.

\subsubsection{Media Source Extensions}

In 2016 the \gls{w3c} enhanced the capabilities of browser media pipelines by introducing the \gls{mse} standard \cite{TODO}. With this extension, a JavaScript application can dynamically adapt an audio or video stream. According to \cite{mse-google}, the main use cases are adaptive streaming, dynamic ad insertion and time shifting. As an alternative to a media stream, the \gls{mse} standard introduces a media source and a source buffer API to the JavaScript media pipeline. A media source can be directly assigned to an \lstinline|<audio>| or \lstinline|<video>| tag and contains a set of source buffers. These buffers can then be filled, sought and flushed to allow seamless playback from multiple stream sources. The \gls{mse} itself does not enforce any kind of codec to promote inter device and browser compatibility.

\begin{figure}
\centering
\includegraphics[width=.5\textwidth]{graphics/media-stream-pipeline.pdf}
\caption{Browser media streaming pipeline and network stack}
\label{fig:pipeline}
\end{figure}

\subsection{WebRTC Media}
\label{subsec:webrtc-media}

"WebRTC provides media acquisition and delivery as a fully managed service: from camera to the network, and from network to the screen", as \citet[\S18.5]{high-performance-browser-networking} describes it. So, for streaming media from one peer to another, the browser provides a powerful, high-level interface to developers. An \lstinline|RTCPeerConnection| as defined by \cite[\S4.4]{webrtc-w3c} can add streams on the producer side as well as receive streams on the consumer side. Under the hood, however, browsers must perform a complex transmission process (see \cref{fig:pipeline}). Because \gls{webrtc} runs on top of \gls{udp}, the connection is inherently unreliable. However, for live streaming applications, timeliness at the cost of video quality is deemed more important than a persistent quality at the cost of delays and stutter \cite[\S18.3]{high-performance-browser-networking}. The \gls{webrtc} network engine uses a stack of two transport protocols to deliver the best possible media stream across a constantly changing connection. \Gls{srtcp} is used to measure and exchange package loss, sequence errors, jitter and more between the two peers. Those statistics influence how the browser chooses appropriate quality and codec settings. \gls{srtp} on the other hand is used to transmit the actual audio and video streaming data.

\subsection{HTTP Streaming}
\label{subsec:http-streaming}

With the adoption of \gls{html} 5, online video streaming has moved from plugin-based approaches like Adobe Flash \cite{adobe-flash} or Microsoft Silverlight \cite{microsoft-silverlight} to native, browser-based technologies. Namely, two competitors have emerged: \gls{hls} and \gls{mpegdash}. The former was developed by Apple, and although not being internationally standardized, \gls{hls} was adopted by some other vendors as well \cite{caniuse-hls}. The successor \gls{mpegdash} has been standardized by \citet{iso-mpeg-dash} and does not require specific video codecs and has built-in \gls{drm} capabilities. What both have in common, is that they split media streams into short segments and use a manifest file to stitch them back together on the client side. This, plus the fact that they run on top of HTTP, make them work well on modern \glspl{cdn}, as \cite{hls-vs-dash} points out.

\subsection{Media Formats}

When considering the fragmented landscape of browser media streaming technologies, the actual media encodings being used are even more vendor and version dependent, as the compatibility table for in-browser playback at \cite{media-format-browser-compat} shows. For playback, common ground between vendors can be found with VP8 compression in a WebM container or H.264 in MP4. However, when trying to encode media recorded on-device, the set of supported formats is even smaller and, according to Phillips, the broadest interoperability can be achieved with H.264 \cite[\S5.1]{webrtc-hacks-safari}. On the audio side, more modern and effiecient formats like Opus and Vorbis have emerged. However, the widest support is still enjoyed by traditional MP3 \cite{media-format-browser-compat}.

\newpage

\section{Video Streaming Applications}

Efficiently delivering streaming video over the internet has long been an active research topic. Non-linearity, rich user interactions and diversification of media sources have driven users from consuming terrestrial television or hard copy video to internet based media platforms.
These platforms and their underlying technologies can be categorised by (1) offered media types and (2) content distribution architecture. Some platforms offer pre-produced videos, be it professional movies or amateur vlogs, to be streamed on demand (\gls{vod}). Other platforms offer live video streams with realtime viewer interactions.
Both type of video offerings can be found to be built upon a centralised or decentralised streaming architecture. \vref{fig:media-platforms-quadrant} shows examples of platforms arranged by category.

\begin{figure}
\centering
\includegraphics[width=.5\textwidth]{graphics/media-platforms-quadrant.pdf}
\caption{Examples of media platforms by delivery and content category}
\label{fig:media-platforms-quadrant}
\end{figure}

\subsection{Centralised Streaming}

The most popular video platforms, YouTube and Netflix, currently account for TODO percent of internet traffic \cite[p 18]{phenomena-report}. Dealing with such extreme user numbers, centralised video delivery architectures are complex and costly \cite{market-driven-p2p}. To deliver compelling \gls{qos} to users, platforms must ensure their networks can retrieve content from a live source or their catalogue with low latency and deliver it in a format optimal for the end users' device and network capabilities. There must be a global server infrastructure for storing, caching and delivering content. Platforms that allow user-generated content must also provide ingress servers, that accept video uploads from users.

These expenses drive platform providers to heavy use of video advertising or subscription models.

\paragraph{Ingress}
Platforms with user-generated content have grown in popularity and significance and, according to \citet{twitch-case}, can be characterised by opening content creation to all of their users and allowing video consumers to give realtime feedback in the form of chats or likes. This interactivity poses tight constraints on latency, or users can become frustrated \cite{TODO}. Periscope, for example, constraints its chat function to the first 100-200 viewers per stream \cite{TODO}. These early participants will be provided with a low latency \gls{rtmp} video stream \cite{TODO}. Subsequent viewers will be served an \gls{hls} video from a global \gls{cdn}. The latter introduces two further sources of delay \cite[\S2-3{periscope-experience}: 1) \gls{hls} segments the video into smaller files, so the duration of such a chunk adds to overall latency. 2) According to network analysis by \citet{periscope-experience} Periscope uses different \textit{Cloud Computing} providers for ingress and \gls{cdn} use.

Twitch, for example, reserves computationally expensive trans-coding of live streams for its premium users \cite[\S2]{twitch-study}. That means, viewers of free channels can only receive the video in the format it was originally uploaded in. Since, unlike Periscope, Twitch also offers access to streams through their website \cite{TODO}, this excludes users from content in formats incompatible with their browser. If a channel subscribes to their premium service trans-coding introduces a delay of up to 10 seconds \cite[\S4.2]{twitch-study}.

\citet[III A.]{content-harvest-network} argue, that the so-called \textit{first mile} between the live video producer and the systems ingress server bears a significant optimisation opportunity. Because upload may be hindered by unstable network conditions, any delay directly impacts the \gls{qoe} of all viewers of the stream.

\paragraph{Transcoding}
SFU or whatevs, Transcoding units

\paragraph{Quality of Experience}
\citet*{personalized-live-streaming-experience} propose schemes to optimise variable video bitrates to improve the perceived \gls{goe} for viewers.

\paragraph{Delivery}
Because HTTP streaming protocols are not built for multicasting, even popular live streams, that many users view simultaneously, have to be delivered to each device individually. Therefore, providers use multiple server farms \cite{TODO-network-book} in strategic geographic locations of their user market. Streams must be routed intelligently from ingress to egress locations within a provider's network. Predicting a streams popularity and allocating resources in locations geographically close to viewer demand can help reduce latency \cite{TODO}.

According to \citet{TODO}, the strategy of Twitch is to serve content from the vicinity of every \gls{ipx}. Social photo and video sharing site Instagram, on the other hand, goes even further and serves content from locations in most major cities of its user market \citet{TODO}. These efforts to bringing server resources as close to users as possible are often referred to as edge or fog computing \cite{fog-computing, object-store-fog-edge-ipfs}.

Despite all efforts, especially less financially secured platforms struggle to come up with enough resources to deliver \gls{qos} without compromise.
...

- Cisco says mobile video streaming is lit

\paragraph{Social Aspects}
censorship vs piracy

\begin{figure}
\centering
\includegraphics[width=.5\textwidth]{graphics/streaming-types.pdf}
\caption{Centralised versus decentralised video streaming architecture}
\label{fig:pipeline}
\end{figure}

\subsection{Decentralised Streaming}

In contrast, decentralised video streaming technologies build multicast streaming on top of an overlay network in the application space. These overlays can be categorised as tree or mesh shaped or a combination thereof. Each peer becomes part of the network and has to assist in its maintenance and content delivery. Decentralised streaming approaches need to handle peers joining and leaving the network, called \textit{peer churn} \cite[\S7.5]{tanenbaum_wetherall_2011}. Like centralised platforms, they must also cope with varying network performance of its users. However, upload capacity becomes a important factor for every peer, not just content producing ones. This means, that finding a compromise between the depth of the tree or mesh structure and the amount of outward connections each node has to maintain is crucial for the system's performance \cite[\SIII.A]{multicast-problems}.

\paragraph{Content Discovery}
In fully decentralised approaches, content discovery is another issue to consider. Whereas centralised media platforms can index, aggregate and even recommend content to users by consulting their databases, decentralised applications lack such central look–up tables by design. Information about peer and content availability must be propagated through the network, creating further traffic and computational overhead.

\paragraph{DONet}
DONet, which was first implemented in the widely researched CoolStreaming, was introduced by \citet{coolstreaming}. It constructs a mesh of peers gossiping (\ref{gossiping}) about available content, members of the network and direct connections. Each peer maintains an incomplete view of the network and a set of peer partnerships. Video streams are split into segments of a defined length and the segments are broadcasted through the network individually. Nodes request segments from their partners, cache segments they receive and add them to their own playback buffer \cite[\SIII.B]{coolstreaming}. When a node detects a stalling upstream partner, it tries to find a better source \cite[\SIV.A]{coolstreaming-design-theory}.

\paragraph{AnySee}
Mesh–based streaming networks like CoolStreaming can achieve good utilisation of peer bandwidth and discover fresh peers with low overhead \cite[\SII]{anysee}. However, \citet[\SIII]{anysee} find that, "Due to the random selection algorithm, the quality of service cannot be guaranteed, such as the startup delay". AnySee evolves \gls{p2p} streaming nodes with a set of manager components \citep[\SIII.B-F]{anysee}. 1) The \textit{Mesh-based Overlay Manager} actively tries to align the mesh network with the physical network by flooding its neighbourhood with \gls{ttl}–bound messages and re–aligning overlay connections. 2) The \textit{Single Overlay Manager} measures and propagates a nodes time offset from the video source and ensures the streaming path is arranged according to this offset. 3) The \textit{Inter-overlay Optimization Manager} secures a set of backup streaming paths through a delay–based tracing algorithm. 4) The \textit{Key Node Manager} queues requests and prioritises stream requests from other peers. 5) The \textit{Buffer Manager}'s task is to request upstream video and providing it to the applications \gls{ui}.

+ continuouity index
+ buffer size 120 vs 40 seconds
- timestamps be shitty

Further, AnySee introduces location awareness to encourage peer partnering close to their geographical vicinity \citep[\SV.B.1]{anysee}. In their initial implementation, this was achieved by matching \gls{ip} addresses of their research network (\gls{cernet}) to known building locations.

\paragraph{HLPSP}
Another refinement to CoolStreaming was introduced by \citet*{hlpsp} called \gls{hlpsp}. This approach creates a layered hierarchy in which peers with higher uploads are positioned in levels closer to the video source. Peers then serve other peers in their own or subsequent levels \cite[\S3]{hlpsp}. The gauging of bandwidth is left up to the network entry point (also referred to as signal server or tracker). This central server assigns a level to a newcomer and provides a list of possible connection partners. If not enough suitable partners are found, the tracker scans the hierarchy and downgrades peers to free capacity on the newcomers level \cite[\S3.3.2]{hlpsp}. This results in efficient bandwidth utilisation, but only accounts for single source scenarios.

- routing
- overlay creation -> timetofirstbyte
- ...problems of streamer cite

\newpage
\input{content/2-background/2-network-topolgies.tex}
\newpage
\section{Mesh Algorithms}
In a decentralised network...

\newpage
TODO:
- streaming:
    - browser media
    - live streaming
        - architectures
        - periscope/twitch
- network: 
    - osi
    - overlay networks
    - browser networking: webrtc ice,stun,turn,simplepeer
    - mesh: OSLR, BATMAN
    - content addressing vs location addressing
- p2p
    - allgemein vorteile/problematiken
    - voll dezentral vs semi-dezentral
    - addressing vs searching
        - pulsarcast
    - exisiting technolgies: bittorrent , GNUTELLA/Fasttrack, IPFS
    
%\chapter{Design}
\section{Design Considerations}

Up to this point, the reader has learned about multiple popular video streaming applications. There is a wide range of specialisations, ranging from gaming\footnote{Twitch. URL: {https://twitch.com}}, lip–syncing\footnote{TikTok. URL: {https://www.tiktok.com}}, social media\footnote{Houseparty. URL: {https://houseparty.com}} to conferencing\footnote{appear.in. URL: {https://appear.in}}. In a different sector, \gls{b2b} services offer complete video streaming solutions to companies and developers. They provide \glspl{sdk}\footnote{SimpleWebRTC. URL: {https://www.simplewebrtc.com}} and host all necessary infrastructure for video delivery and encoding\footnote{Bambuser. URL: {https://bambuser.com}}. What these offerings have in common, is that they rely on server–client infrastructure. User generated content is uploaded to servers, processed and distributed over \glspl{cdn}. Commonly, they use \gls{hls} and \gls{mpegdash} streaming to feed live video to clients. Even though their marketing strategies and business models vary, their technological requirements boil down to two categories: 1) few channels broadcasting to large audiences and 2) smaller groups where multiple users contribute a live feed.

In academia, there have been numerous efforts to move towards \gls{p2p} streaming networks \cite{anysee, coolstreaming, hlpsp}. Although proven to be applicable \cite{skype-p2p-primer, tox-chat-app}, these technologies are scarcely found in mass market products today. The reasons for that are manifold and debatable – ranging from concerns about quality of service \cite{skype-ditching-p2p} to the tainted reputation of \gls{p2p} in the face of illegal torrenting \cite{p2p-social-impact}. One further, yet prominent reason, this thesis is trying to challenge, is \textit{platform–dependency}. Many prior \gls{p2p} streaming implementations are requiring native applications as they need access to the \gls{ip} network stack. As browser \glspl{api} have matured, it has become feasible to build a platform–independent streaming network with \gls{js} and \gls{webrtc} connections. With the browser as its run–time, it will function mostly independent of device category or operating system. For the user, this lowers the barrier to join the network and for developers, this simplifies reaching audiences across device categories.

To 

- advances in p2p and mesh networks
- kademlia
- torrent

- ipfs is also running on in the browser
- ipfs is good as a distributed file system but not suitable for realtime data
- webrtc muy complicado

This thesis has set out to develop a live video streaming application
When developing a decentralised \gls{p2p} system, an open approach to software development and system design can only help the project succeed.

\newpage
\section{Requirements}

\begin{itemize}
    \item Browser restrictions / Max defined connections / Client strebt connections an
    \item WebRTC connection negotiation restricition. Connection kann nur über mittelsman aufgemacht werden
    \item Jeder Client soll dezentralen zustand anstreben -> Connection zu zentralen komponentnen lösen (Signal)
    \item Churn Restistance / Self healing
    \item Gobbel detetection wenn client verbindung zum "hauptnetzwerkverliert" soll dies erkannt werden
    \item Baustein/SDK mit einfacher schnittstelle um usecase unabhängig sein (Transparencey)
    \item Deterministische Simulationsumgebung um algorithmus verifizieren zu können
    \item So nah wie möglich an real time
    \item Modularer Codeaufbau um einfache Erweiterung in Zukunft zu ermöglichen (Erweiterung von Transportlayern / Rollen / MessageTypes)
    \item Unterstützung von verschiendenen Tranportlayern sodass algorithmus connection layer agnostisch ist (WSS/WebRTC/Bluetooth/...)
\end{itemize}

\newpage
\section{Scope}

\begin{itemize}
    \item Scope
    \begin{itemize}
        \item Nutzer sollen Video an X andere nutzer streamen können
        \item Clients sollen nicht alle anderen clients im netz kennen. Broadcast "flooding" reicht
        \item Es soll mehere Räume abegildet werden können / multiple one-2-many scenarios
        \item Es soll immer nur ein Nutzer Streamen können pro Raum
        \item Security soll abgebildet werden können aber für prototyp nicht implementiert werden
    \end{itemize}
    \item Beschreibung des Algorithmus
    \begin{itemize}
        \item Network boarding process
        \item Peer Discovery
        \item Content Discovery / Channel Discovery
        \item Rollenkonzept vorstellen
        \item Multi-Agentensystem / Convergence 
    \end{itemize}
\end{itemize}

\newpage

%
\chapter{Implementation}
\begin{itemize}
    \item Architektur (Mitosis/Sim/Viz/CLI)
    \item Mitosis
    \begin{itemize}
        \item Address
        \item Roles
        \begin{itemize}
            \item Rolemanager
            \item Roles
        \end{itemize}
        \item Peermanager
        \item Peertable/Conenctiontable
        \item Messages
        \begin{itemize}
            \item Message broker
            \item Message types
        \end{itemize}
        \item Clock
        \item Vorstellung der Rollen
        \item Connections
        \begin{itemize}
            \item Unterschiedliche Connections vorstellen
            \item Connection aufbau
            \item Metering
        \end{itemize}
    \end{itemize}
    \item Simulation
    \begin{itemize}
        \item Instructions
        \item Scenarios / Scenarioparser
        \item MockConnections
        \item One clock to rule them all
    \end{itemize}
    \item Visualiserung
    \begin{itemize}
        \item D3
        \item Vorstellung der UI (peer-tab/message/log/stats)
    \end{itemize}
    \item CLI
    \begin{itemize}
        \item Run simulation in node
        \item Automate parametrized benchmarking
        \item Export results in csv/json
        \item Plot results of svg/json
    \end{itemize}
    \item Symbiosis
        \item Vorstellung der UI
\end{itemize}

\subsection{Architecture}
Based on the design and the the self imposed requirements an architecture has been designed how all the components shall work together.
First of all an over all architecture has been designed. The architecture defines the high level modules and how they should interact with each other. 
In the next step each module got their own architecture.

\subsubsection{Overall Architecture}
The overall architecture consists of the following modules
\begin{itemize}
    \item Mitosis-Core
    \item Simulation
    \item Visualisation
    \item \gls{cli}
    \item Symbiosis
\end{itemize}

Mitosis-Core is the main component which sets up the Peer-To-Peer mesh network. 

\newpage

%\chapter{Analysis Tools}
In a distributed system the interaction between clients plays an important role. Setting up scenarios with real clients becomes unfeasible due to the complexity that a distributed system brings with it. To analyse and debug the state or a problem of a client, a deterministic environment is required, where a state or a problem can be easily reproduced.
Thus, simulation tools are needed where multiple virtual instances can be spawned and monitored easily. The virtual instances have to interact with each other like like they would do in a real environment. Also, the virtual instances shall provide information, e.g. data that is sent, without modifying the actual implementation. Last but not least, the virtual instances shall live in a controllable environment where all instances can be accessed, started and stopped at the same time. 

This chapter describes the tools that have been developed to analyse and debug the implementation of Mitosis. \vref{fig:anl-overall-stack} presents the stack of the developed tools.

Each component is described briefly in the following before going further into details.

\begin{itemize}
    \itembf{Simulation} The core component that spawns virtual instances and runs Mitosis in each of it. The communication between the virtual instances is mocked, thus a physical connection is not required. The virtual instances can be orchestrated through a \textit{Scenario} that the Simulation is executing.
    \itembf{Visualisation} Visualises the output that is provided by the Simulation component. Also allows interaction with each virtual instance. Inspector tools allow to inspect the state of each instance. The virtual instances are visualised as a force directed graph.
    \itembf{CLI} Executes the simulation in a Node.js environment and does not have any graphical feedback. Parameterized execution of scenarios allows benchmarking. Results are exported as files as soon as the execution is done.
    \itembf{CLI-Import} Parts of the results of the CLI can be imported so the result can be restored as a force directed graph and further analysed.
    \itembf{Realtime Visualisation} A dedicated test application that executes Mitosis in a real environment reports its state to a server. The state of each online client and the connections between them are visualised in almost realtime. 
\end{itemize}

\begin{figure}
\centering
\includegraphics[width=0.7\textwidth]{graphics/analysis-tools/analysis-tools.pdf}
\caption{Stack}
\label{fig:anl-overall-stack}
\end{figure}
\newpage
\section{Simulation}
The simulation has to fulfil several self–imposed requirements:
\begin{itemize}
    \item Run multiple virtual instances of Mitosis
    \item Allow orchestration via scenarios
    \item Mock physical connections
    \item Allow freezing and unfreezing the whole simulation
    \item It shall be easy to setup a scenario
\end{itemize}

\paragraph{Virtualisation}
The simulation spawns multiple virtual nodes, with each node behaving like it would in the real world. Each node is running a Mitosis instance and the amount of nodes to spawn is defined in a \textit{Scenario}. 

Nodes communicate via a \lstinline{MockConnection} with each other. To keep it close to the real world, each node can have different connection settings like latency and stability.
In order to get deterministic results the simulation is using a pseudo–random function that is initialised with a seed.

\paragraph{Orchestration}
The simulation parses a scenario on initialisation. A scenario specifies the parameters of the simulation and can have multiple instructions. Via the parameters, the default parameters of Mitosis can be set e.g. \lstinline{DIRECT_CONNECTIONS_MAX}, but also the lifetime and network settings of a node.

Instructions are tasks that the simulation has to execute. Each task can register for execution at a certain point in time.
The simulation maintains its own clock which is provided by the Mitosis package (cf. \vref{sec:mit-clock}).
When the point in time is reached, the instruction is executed. All instructions for the given point in time are executed, ordered by the registration order of the scenario.

The task of an instruction can be anything from adding one node to adding multiple nodes in an interval. Several instructions have been implemented:

\begin{itemize}
    \itembf{AddPeer} Adds a new node with specific configuration
     \itembf{RemoveConnection} Destroys a configurable connection
    \itembf{RemovePeer} Removes a configurable node
    \itembf{Clock instructions} \lstinline|pauseClock| to pause the clock of the simulation, \lstinline|startClock| to continue, \lstinline|setClockSpeed| to set its speed and \lstinline|stopClock| to finish the simulation
    \itembf{Generative instructions} \lstinline|generatePeers| to generate node continuously with a configurable upper bound and interval, \lstinline|eliminatePeers| to destroy random nodes in an interval to simulate Peer-Churn
\end{itemize}

\paragraph{Mock Connections}
The simulation makes use of the flexibility of the Mitosis \lstinline|ProtocolMap| to override the existing implementations for given protocols (cf. \vref{sec:mit-connections}). The implementations are overriden with a  \lstinline|MockConnection|. Instead of using the implementation to communicate via WebSocket and WebRTC, the MockConnection delegates the transport to the simulation. The simulation is directly accessing the receiver node and calls its receive method while considering both the sender's and the receiver's network configuration. The network configuration specifies the transmission delay and can also lead to a drop of a message when the stability is set to less than $\ 100\% $.

\paragraph{Freezing the Simulation}
One import important feature, that is really useful when debugging the system, is freezing the state of the nodes. During a freeze, all nodes stop executing their actions. When the simulation is unfrozen, they continue with the execution.

To allow freezing, the simulation is creating its own global clock (cf. \vref{sec:mit-clock}). Mitosis allows to pass a clock into the constructor, which is then used for the execution cycle. By passing a fork of the global clock into the Mitosis constructor, each instance is synchronised with the global clock. When the global clock is paused, each instance is paused as well.
Also, the clock speed can be controlled globally. When the global clock ticks faster, all Mitosis instances are ticking faster as well. This allows fast forwarding, slowing down or even executing one tick at a time.

\paragraph{Creating a new scenario}
Creating a new scenario is fairly easy. The format of a scenario is JSON which \say{is an open-standard file format that uses human-readable text}\cite{wiki:json}.
Simulation parameters and instructions are specified in the scenario. The simulation is parsing the file during initialisation and executes the instructions.

A basic Scenario is presented in \vref{lst:anl-scenario} which sets up two peers by using the \lstinline|addPeer| instruction. Both peers are configured with a unique address. One peer is also configured with the default role \signal which is a basic requirement for the boarding process. 

\newpage
\section{Visualisation}

\begin{figure}
\centering
\includegraphics[width=1\textwidth]{graphics/analysis-tools/visualisation-overlay.jpg}
\caption{User interface of the Visualisation}
\label{fig:anl-visualisation}
\end{figure}

The Visualisation is build on top of the simulation and is presented in \vref{fig:anl-visualisation}. It takes the given output from the simulation and visualises its state. It is also using the global clock from the simulation and updates its state on every tick. Also it allows to interact with the nodes, like deleting it from the simulation or change the virtual network settings.

The interface consists of two parts. On the left side all nodes and their connections are represented as a graph. D3.js\footnote{D3.js. URL:{https://d3js.org/}} is used for the graph and further more the d3-force\footnote{D3.js d3-force module. URL:{https://github.com/d3/d3-force}} module is used.

Each node is represented as a circle with the node id as text label. Real names instead of pseudo ids were chosen for the node ids, to make it easier to discuss about a scenario. A node has also a colour which represents its role. Black represents a node with the role \signal. Dark grey a node with the role \router and light grey represents a node with the role \peer.  
When a node has a virtual connection to another node, the connection is represented as a grey line.
In the top left corner the current clock tick is displayed with an icon that indicates whether the clock is ticking or paused.

The sidebar on the right allows to interact with the simulation and to inspect a node. The first section allows to interact with the clock, pause, restart and a tick can be triggered. Also the scenario to execute can be chosen.

The section below allows to search for a node in the simulation. When a node for a search term was found it is automatically selected. A selected node is highlighted in the graph view but also the node inspection tool appears.

The inspection tool consists of six tabs and are presented in \vref{sec:visualisation-tabs}: 

Node settings, Peer table, Channel table, Network statistics, Incoming/Outgoing messages and Log entries. 

In the tab \textit{Channel table} a virtual stream can be started. The nodes behave like in the real world and broadcast their stream to other peers. The \glsfirst{dag} that is described in \vref{sec:design-stream-construction} is created. The visualisation is also able to visualise this graph as presented in \vref{fig:anl-sim-stream-active}. The black arrows indicate the direction of the stream from one node to another.


\newpage
\section{CLI}

\begin{figure}
\centering
\includegraphics[width=1\textwidth]{graphics/analysis-tools/cli.jpg}
\caption{CLI}
\label{fig:anl-cli}
\end{figure}

As the visualisation is doing some heavy lifting to visualise the state of each node, it is not really suitable for large scenarios. Also, using it for benchmarking with different parameters is a tedious task. It is very good for analysing and debugging small scenarios, yet it has its limits for performance and scaling analysis.

Thus, a \glsfirst{cli} has been developed which runs the simulation in a headless mode in a Node.js environment. It accepts a path to a scenario JSON as paramter and also accepts a log level. By default the logging is turned off but can be turned on by the log level parameter. The scenario is executed and the results are printed (\vref{fig:anl-cli})and saved as a \gls{csv} file. A JSON file is exported with the current state of the nodes. The file can be imported into the visualisation to visually analyse the state.

Not only is it faster to run a scenario, it also accepts a new type of configuration in the scenario which allows benchmarking. The parameter allows running one scenario in a batch job with different settings for Mitosis. The parameter can be configured in the scenario. When the benchmark is finished the CLI also saves the results as a \gls{csv} and as a JSON file.

\chapter{Analysis}

- Scenarios


- Hard limits
    - ttl
    - stream baum vs minimal spanning tree
        - histogram

- Discussion
- scaling further
    - dht

- nucleus scenarios
    - gobble
    - eclipsed group re–joining
    - router lost
    - glare problem: two peers try to open same connection
        - circuit glare, aka dual seizure like in phone systems
        - see \cite[pp. 194-194]{signaling-systems-book}


Random Quotes:
\say{Analysis of a P2P system by Saroiu et al. [3] show that the longer a node has been up, the more likely it is to remain up for another hour.}(http://gleamly.com/article/introduction-kademlia-dht-how-it-works). IPFS bitswapping auch da hinschreiben

\section{Analysing small scenarios}
\subsection{Genesis}

\begin{figure}[htb!]
  \centering
    \subfloat[]{\includegraphics[width=0.33\textwidth]{graphics/analysis/mini-scenarios/become-router/1.png} 
    \label{fig:filmstrips-genesis-a}}
    \subfloat[]{\includegraphics[width=0.33\textwidth]{graphics/analysis/mini-scenarios/become-router/2.png} \label{fig:filmstrips-genesis-b}}
	\subfloat[]{\includegraphics[width=0.33\textwidth]{graphics/analysis/mini-scenarios/become-router/3.png} \label{fig:fig:filmstrips-genesis-c}}
	\caption{Join network as first peer}
\label{fig:overlay-topologies}
\end{figure}
\subsection{Join network}

\begin{figure}[htb!]
  \centering
    \subfloat[]{\includegraphics[width=0.25\textwidth]{graphics/analysis/mini-scenarios/join-network/1.png} \label{fig:filmstrips-join-a}}
    \subfloat[]{\includegraphics[width=0.25\textwidth]{graphics/analysis/mini-scenarios/join-network/2.png} \label{fig:filmstrips-join-b}}
	\subfloat[]{\includegraphics[width=0.25\textwidth]{graphics/analysis/mini-scenarios/join-network/3.png} \label{fig:filmstrips-join-c}}
	\subfloat[]{\includegraphics[width=0.25\textwidth]{graphics/analysis/mini-scenarios/join-network/4.png} \label{fig:filmstrips-join-d}}
	\caption{Join network as second peer}
\label{fig:overlay-topologies}
\end{figure}
\subsection{Join network but router is already full of capacity}

\begin{figure}[htb!]
  \centering
    \subfloat[]{\includegraphics[width=0.33\textwidth]{graphics/analysis/mini-scenarios/router-full-redirect/1.jpg} \label{fig:filmstrips-redirect-a}}
    \subfloat[]{\includegraphics[width=0.33\textwidth]{graphics/analysis/mini-scenarios/router-full-redirect/2.jpg} \label{fig:filmstrips-redirect-b}}
	\subfloat[]{\includegraphics[width=0.33\textwidth]{graphics/analysis/mini-scenarios/router-full-redirect/3.jpg} \label{fig:filmstrips-redirect-c}}
	\caption{Join network peer}
\label{fig:overlay-topologies}
\end{figure}
\subsection{Bottleneck prevention}

\begin{figure}[htb!]
  \centering
    \subfloat[]{\includegraphics[width=0.33\textwidth]{graphics/analysis/mini-scenarios/bottleneck-prevention/1.jpg} \label{fig:filmstrips-bottleneck-prevention-a}}
    \subfloat[]{\includegraphics[width=0.33\textwidth]{graphics/analysis/mini-scenarios/bottleneck-prevention/2.jpg} \label{fig:filmstrips-bottleneck-prevention-b}}
	\subfloat[]{\includegraphics[width=0.33\textwidth]{graphics/analysis/mini-scenarios/bottleneck-prevention/3.jpg} \label{fig:filmstrips-bottleneck-prevention-c}}
	\caption{Preventing a bottleneck}
\label{fig:overlay-topologies}
\end{figure}
\subsection{Send message}

\begin{figure}[htb!]
  \centering
    \subfloat[]{\includegraphics[width=0.25\textwidth]{graphics/analysis/mini-scenarios/send-message/1.jpg} \label{fig:filmstrips-send-message-a}}
    \subfloat[]{\includegraphics[width=0.25\textwidth]{graphics/analysis/mini-scenarios/send-message/2.jpg} \label{fig:filmstrips-send-message-b}}
	\subfloat[]{\includegraphics[width=0.25\textwidth]{graphics/analysis/mini-scenarios/send-message/3.jpg} \label{fig:filmstrips-send-message-c}}
	\subfloat[]{\includegraphics[width=0.25\textwidth]{graphics/analysis/mini-scenarios/send-message/4.jpg} \label{fig:filmstrips-send-message-d}}
	\caption{Path of a message}
\label{fig:overlay-topologies}
\end{figure}
\subsection{Broadcast}

\begin{figure}[htb!]
  \centering
    \subfloat[]{\includegraphics[width=0.25\textwidth]{graphics/analysis/mini-scenarios/stream/1.png} \label{fig:filmstrips-broadcast-a}}
    \subfloat[]{\includegraphics[width=0.25\textwidth]{graphics/analysis/mini-scenarios/stream/2.png} \label{fig:filmstrips-broadcast-b}}
	\subfloat[]{\includegraphics[width=0.25\textwidth]{graphics/analysis/mini-scenarios/stream/3.png} \label{fig:filmstrips-broadcast-c}}
	\subfloat[]{\includegraphics[width=0.25\textwidth]{graphics/analysis/mini-scenarios/stream/4.png} \label{fig:filmstrips-broadcast-d}}
	\caption{Broadcasting tree creation}
\label{fig:overlay-topologies}
\end{figure}
\newpage
\input{content/6-analysis/2-meshing.tex}
\newpage
\section{Streaming Analysis}

\subsection{Topology}

Since video streams over \gls{webrtc} connections decrease in quality per hop, as discussed in \vref{TOOD}, an optimal streaming topology would would be a \textit{perfect k–ary tree}, given nodes can upload $k$ streams simultaneously. A perfectly balanced binary tree, for example, would be the optimal streaming topology for $k=2$.

As the streaming topology is built on the collective knowledge of the overlay network, it depends heavily on the state the overlay mesh has reached at the point the stream is started. It must also be considered, that the streaming tree is constructed in the two phases detailed in \vref{TODO}. The push phase aims at achieving availability fast and does not regard tree balance.
It is also important to note, that due to node churn, the stream topology can change unexpectedly as nodes try to re–acquire a connection to a source provider.

The quality of the streaming topology is inspected in a churn–free simulation setup. Given a scenario of $n=100$ nodes and $t=1500$ ticks a stream is automatically started at $t_{500}$. The network constructs its stream topology and the active stream connections are queried at the end of the scenario. The topology is traversed and the distance to the stream source is extracted from every node. For simplicity, the connections are handled unweighted and just the hop count is considered.
Using different random seeds, multiple runs are averaged using the mean node count per distance.

\vref{fig:streaming-topology-histogram} shows the results of the hop distribution for streaming topologies for $k=2$ and $k=3$. For comparison, the histogram also shows the distribution in an optimal binary and ternary tree given $n=100$ nodes.

\begin{figure}
\centering
\includegraphics[width=1\textwidth]{graphics/analysis/streaming-topology-histogram.pdf}
\caption{Stream consumers per distance from stream source}
\label{fig:streaming-topology-histogram}
\end{figure}

It is evident, that the trees constructed by the overlay mesh are far from the perfectly balanced trees. A binary tree with $n=100$ nodes would not exhibit hop distances beyond $6$ and a ternary tree would not exceed distances of $4$. The respective streaming trees show a much broader distribution. The spikes caused behind the quadratic/cubic function of the optimal trees, are not to be found in the stream topologies. Yet, half the nodes exceed $8$ connection hops for $k=2$ and half the nodes for $k=3$ are above $6$ connections away from the source.

This non–optimal distribution can be explained by collisions during the push phase. As multiple nodes in the same radius around the stream source receive the channel, they try to push it out to their most reliable neighbour. However, chances are, their own best neighbour is also the choice of some other node on the same radius. As a result, this popular choice receives two connection offers, accepts only one and leaves the other offerer as a leaf node in the tree. While this is not a desirable state, the topology can self–heal during the pull phase. As nodes join or become interested in the channel, they will seek providers and can attach to the aforementioned leaf node.

So, in a second experiment, a more realistic scenario is set up. Nodes are not instantiated at once, but keep joining at a rate of $1/t$ to a total number of $n=50$. After half the nodes have joined the mesh, the stream is automatically started. \vref{fig:streaming-topology-continuous-histo} shows how the hop distances are distributed when nodes are continuously joining instead of being initiated before the stream start.

\begin{figure}
\centering
\includegraphics[width=1\textwidth]{graphics/analysis/streaming-topology-continuous-histo.pdf}
\caption{Stream consumer distance in continuous join rate scenario}
\label{fig:streaming-topology-continuous-histo}
\end{figure}

In summary, the streaming topology is heavily dependent on the underlying mesh size and structure. To cope with larger numbers of stream consumers, the system would have to perform tree optimisation. Whether this is feasible, however, depends on the application dynamics. If video channels are long–lasting and audiences are large, it would be worthwhile. If video channels are only alive a few minutes and audiences are usually small, tree optimisation would not be worth the extra negotiation effort.

\newpage
\input{content/6-analysis/4-security.tex}
%\chapter{Résumé}
\label{chap:resume}
\section{Future Work}
The implementation of the proposed design is working and it can already handle quite a lot of nodes as the analysis part has shown. Due to the lack of more time not everything that has been proposed in the design chapter could be implemented. Based on the insights of the analysis also some parts of the design have to be improved.
Besides optimising the implementation and design, future work also proposes possible extensions of the \gls{Mitosis} SDK.

\paragraph{Multi Cluster Support}
The implementation has shown how one cluster can scale up and how it behaves under various conditions. Yet, a cluster can only absorb a certain number of peers. The final size can be controlled the \gls{ttl} of the \routerAlive message and the connection limits. In order to scale beyond the capabilities of a single cluster, the implementation has to support the coexistence of multiple clusters. There must be mechanisms for creating clusters, for closing abandoned clusters and for load balancing between them. The proposed design and implementation have been referred to as \gls{Mitosis}: the envisioned process of dividing clusters into two and reassigning peers based on router proximity. This process has not been included in the implementation yet, but poses a challenging task for future research in this field.

\paragraph{Inter Cluster Communication}
When having multiple clusters the communication from on cluster to another becomes a challenge. One possible approach to solve inter cluster communication would be to introduce a \glsfirst{dht} based on Kademlia. Each node would need a unique identity. When a node enters the network it is assigned to a router. The id of the node would be stored in the Kademlia \gls{dht} as a key and the id of the router would be the value belonging to the key. 

Mitosis already ensures that a peer can always exchange messages with the router it is assigned to. Also it ensures that a router knows all nodes within its own cluster. 

Thus when a peer would want to communicate with a peer from another cluster it would address its message first to its router. As the peer is not in the cluster of the router, the router would then make a key look up in the Kademlia \gls{dht} for the key of the addressed peer id. Through the XOR-Metric the look up would return the value for the given key when the peer exists somewhere in the network. The value would be the id of the responsible router for the addressed peer. 
The router would then make a node look up to find the router that has been returned by the key look up. The message is then redirected to the router that is returned by the node look up. This router in turn can then send the message to the addressed peer as it has to exist within its cluster. 
Example: A message from a node in cluster $\ X $ $\ Node A_X $ to a node in cluster $\ Y $ $\ Node B_Y $ would be routed as:
$\ Node A_X \rightarrow Router_X \rightarrow Router_Y \rightarrow Node B_Y $
    
\paragraph{Improving the peer selection metrics}
Each peer is acquiring new peers in order to full fill its connection goal. For the acquisition process a peer select suitable peers from its direct neighbours and ranks them based on a variety of metrics. The ranking of the peers has a significant influence on the stability of the whole network. But not only does it affect the overlay network but it is also lays the foundation for the streaming quality. The better the peers are selected the better the streaming quality can be. 
WebRTC provides an extensive statistics API to gather information about codec information, dropped frames and jitter of an ongoing streaming session between to peers. Those information could be used as an extension of the existing metrics of Mitosis. 
But not only connection related could be improved also other metrics like uptime and reliability of a client could be added as metrics.

\paragraph{Improving the streaming graph}
To propagate the live stream through the network Mitosis has two phases: phase 1 where the stream is actively pushed from one peer to its two best neighbours and phase 2 where clients who haven't been selected pull the stream from peers that are providers.
However, results of the analysis have shown that the push phase is too aggressive and therefor it results in an unbalanced sub optimal tree. For the quality of the stream it is better to be as close to the source as possible.
Thus the push phase has to be limited to a certain extend in order to get a better tree.
Also the peer-churn problem it not solved optimal. Peers are recovering from a leaving node but it takes time until the stream connection to another provider is established. As an improvement fallback connection have to be established to a sibling of the parent node to allow switching of the stream for a minimal disruption of the playback.

\paragraph{Security Aspects}
As discussed in \vref{sec:security-integrity}, the current implementation of Mitosis does not warrant integrity of node identities. For use in a publicly released application, the node identity creation would have to be secured with a cryptographic proof of work to protect against spamming and flooding. This would also incentivise nodes to hold on to their identity and would allow nodes to build up a reputation within the network. Nodes could then use their persistent identity to secure \gls{dtls} connections. Given certainty on the communication partner's identity, applications could integrate private messaging and private video channels with end–to–end encryption. The persistent identity would be also a must have requirement for the Inter Cluster Communication approach with Kademlia.

\paragraph{Improved Signalling}
Currently there is only one signal server and the current implementation only allows one router. In a multi router scenario the signal would have to deal with load balancing. But not only load balancing becomes interesting but also geografically local awareness. In a optimal scenario the signal would put all geographically nearby nodes into one cluster. The geographical location of a node can be obtained by the signal server through the ip-address of the node.
Also to improve the ingress rate of peers into the network, the implementation should support multiple signal servers. This would also improve fault tolerance and would make it harder to shut down the network.

\paragraph{Media Persistence}
For use cases where live streaming is not enough and the content has to be persisted support for the \glsfirst{ipfs} could be integrated. By using \gls{ipfs} the distributed paradagim would not be interfered but only extended with a distributed persistence layer. 
By using the MediaStream Recording API of the browser \cite{media-recorder-api} the stream can be chunked into segments and each segment can be added into \gls{ipfs}. In fact, IPFS provides a working example how IPFS could be use to realise a \gls{vod} streaming application using their protocol.
\newpage
\section{Conclusion}
We started out with a simple experiment, manually connecting devices using \gls{webrtc} and instructing them to distribute a live video stream.
Our goal was to automate the process and build a system of self–organising agents, that could grow beyond the limits of a fully connected mesh.
We have looked into existing research and implementations of scalable mesh networks, common practices for distributed routing and approaches for live streaming in \gls{p2p} networks.

Our result is the design of a \gls{webrtc}–based overlay network with a custom routing protocol that allows distributing a live video feed by dynamically setting up a \glsfirst{dag} of stream connections. Nodes in the network act as independent agents and communicate with other nodes in their vicinity. Our implementation is inspired by the \glsfirst{bdi} model, as each node has the desire to find optimal connections and thereby strengthens the overall network.

We have verified our implementation by creating our own analysis tools. These helped in testing micro and macro scenarios, ingress of new peers and simulating network weak points. We also inspected the connection limits of our network and analysed the streaming topology. The simulation with its visualisation interface and benchmark \gls{cli}, has greatly helped in design, implementation and analysis of the entire system. Finally, it was possible to test how the implementation behaves under close to real–world conditions.

The results of our simulations have suggested that the system can serve one–to–many streaming scenarios for audiences with hundreds of nodes. A sample application (\vref{sec:symbiosis}) using the \textit{Mitosis} \gls{sdk} was implemented as a proof–of–concept. Preliminary tests with up to 25 real devices have shown stable video feeds under optimal network conditions. Real–world performance with greater variance in devices and higher user numbers has yet to be confirmed.

Both, simulations and real–world tests, have shown that modern browsers are a viable platform for application–layer multicast. It opens up new possibilities to web application developers, to integrate networking with multimedia \glspl{api} and application semantics. The distributed nature of our network, removes the need for server infrastructures, so independent and non–profit applications could integrate our software. For the user, \gls{mitosis}–based web apps are usable without the need for extra software and video stays in the user–space. By extending the implementation with further security measures, platforms can protect their users and content from the prying eyes of corporations and governments.

The \gls{mitosis} \gls{sdk} allows to build a wide range of application genres. Apart from the obvious video chat or broadcasting platform, gaming, e–learning or other genres are conceivable, especially if they leverage the possibilities of the data channel. We hope to inspire a new generation of distributed media live streaming platforms by contributing our open–source \gls{sdk}.

\newpage

%
%
%\renewcommand{\appendixtocname}{Anhang}
%\renewcommand{\appendixname}{Anhang}
%\renewcommand{\appendixpagename}{Anhang}
\appendix
%\chapter{Appendix}
\section{RemotePeer Meter Metrics}
\begin{Listing}[H]
\begin{lstlisting}[xleftmargin=3em]
lastSeen = 0
for connection of allConnections
  lastSeen = max(lastSeen, connection.lastSeen)
\end{lstlisting}
\caption{Youngest LastSeen of all Connections}
\label{lst:mit-last-seen}
\end{Listing}

\begin{Listing}[H]
\begin{lstlisting}[xleftmargin=3em]
lastSeen = getYoungestLastSeen()
isCrashed = currentTick - lastSeen > GLOBAL_CRASH_THRESHOLD
\end{lstlisting}
\caption{RemotePeer is crashed}
\label{lst:mit-crashed}
\end{Listing}

\begin{Listing}[H]
\begin{lstlisting}[xleftmargin=3em]
qualitySum = 0
for connection of allConnections
  qualitySum += connection.quality
averageQuality = qualitySum / allConnections.size  
\end{lstlisting}
\caption{Average connection quality}
\label{lst:mit-average-connection-quality}
\end{Listing}

\begin{Listing}[H]
\begin{lstlisting}[xleftmargin=3em]
bestQuality = 0
for connection of allConnections
  bestQuality = max(bestQuality, connection.quality)
\end{lstlisting}
\caption{Get best connection quality}
\label{lst:mit-best-connection-quality}
\end{Listing}

\begin{Listing}[H]
\begin{lstlisting}[xleftmargin=3em]
hasProtectedConnection = false
for connection of allConnections
  if connection is protected
    hasProtectedConnection = true
    
hasNotSatisfiedConnectionGoal = 
  reportedDirectConnections < GLOBAL_CONNECTION_GOAL 

isProtected = 
  hasNotSatisfiedConnectionGoal && hasProtectedConnection  
\end{lstlisting}
\caption{Is a protected RemotePeer}
\label{lst:mit-welpenschutz}
\end{Listing}

\begin{Listing}[H]
\begin{lstlisting}[xleftmargin=3em]
allViaCons = peer.getAllViaConnectionsTo(remotePeer)
allDirectCons = peer.getAllDirectConnections(remotePeer)

totalCons = allViaCons  allDirectCons
saturation = 
  (GLOBAL_DIRECT_CON_MAXIMUM - totalCons) / GLOBAL_CONNECTION_GOAL

saturation = limit(0, saturation, 1)
\end{lstlisting}
\label{lst:mit-connection-saturation}
\caption{Connection saturation of a RemotePeer}
\end{Listing}

\begin{Listing}[H]
\begin{lstlisting}[xleftmargin=3em]
punishedConnectionAvg = 
  punishedConnections / totalDirectConnections
\end{lstlisting}
\label{lst:mit-punished-conenctions}
\caption{Punished connections average}
\end{Listing}

\begin{Listing}[H]
\begin{lstlisting}[xleftmargin=3em]
reportQuality = bestConnectionQuality * connectionSaturation
\end{lstlisting}
\label{lst:mit-remote-peer-quality-report}
\caption{Quality of a RemotePeer for PeerUpdate}
\end{Listing}

\begin{Listing}[H]
\begin{lstlisting}[xleftmargin=3em]
quality = avgConnectionQuality * (1 - avgConnectionPunishment)
\end{lstlisting}
\label{lst:mit-remote-peer-quality}
\caption{Quality of a RemotePeer}
\end{Listing}
\newpage
\section{Symbiosis}\label{sec:symbiosis-soruce-code}

\begin{Listing}[H]
\begin{lstlisting}[xleftmargin=3em]
<!DOCTYPE html>
<html lang="en">
<head>
  <title>Symbiosis</title>
  <style>
    video {
      height: 100vh;
      width: 100vw;
      object-fit: cover;
    }

    button {
      width: 100px;
      height: 50px;
      background: red;
      position: absolute;
      color: white;
      left: calc(50vw - 25px);
      bottom: 50px;
    }
  </style>
</head>
<body>

<video></video>
<button>Record</button>

</body>
</html>
\end{lstlisting}
\caption{symbiosis.html}
\label{lst:symbiosis-html}
\end{Listing}

\begin{Listing}[H]
\begin{lstlisting}[xleftmargin=3em]
import {Mitosis} from 'mitosis';

const videoEl = document.querySelector('video');
const recordEl = document.querySelector('button');

const mitosis = new Mitosis();

const onStreamAdded = (streamEv) => {
  if (streamEv.type === 'added') {
    videoEl.srcObject = streamEv.stream;
    videoEl.play();
  }
};

const startStream = () => {
  navigator.mediaDevices.getUserMedia({
    video: true,
    audio: false
  }).then(
    (stream) => {
      mitosis.getStreamManager().setLocalStream(stream);
    });
};

mitosis
  .getStreamManager()
  .observeChannelChurn()
  .subscribe(
    channelEv => channelEv.value
      .observeStreamChurn()
      .subscribe(onStreamAdded)
  );

recordEl.addEventListener('click', startStream);
\end{lstlisting}
\caption{symbiosis.js}
\label{lst:symbiosis-js}
\end{Listing}
\newpage
\section{Sample Scenario for the Simulation}
\label{sec:visualisation-tabs}
\begin{Listing}[htb!]
\begin{lstlisting}
{
  "configuration": {
    "newbie": {
      "DEFAULT_SIGNAL_ADDRESS": "mitosis/v1/signal/wss"
    }
  },
  "instructions": [
    {
      "tick": 0,
      "type": "set-clock-speed",
      "configuration": {
        "speed": 50
      }
    },
    {
      "tick": 0,
      "type": "add-peer",
      "configuration": {
        "address": "mitosis/v1/signal/wss",
        "roles": [
          "signal"
        ]
      }
    },
    {
      "tick": 0,
      "type": "add-peer",
      "configuration": {
        "address": "mitosis/v1/alice/webrtc"
      }
    },
\end{lstlisting}
\caption{Basic scenario with a Signal and a Peer}
\label{lst:anl-scenario}
\end{Listing}
\section{Visualisation}
\begin{figure}[H]
\centering
\includegraphics[width=1\textwidth]{graphics/analysis-tools/visualisation-sidebar-sim-settings.pdf}
\caption{Simulation settings}
\label{fig:anl-sim-settings}
\end{figure}

\begin{figure}
\centering
\includegraphics[width=1\textwidth]{graphics/analysis-tools/visualisation-sidebar-node-settings.pdf}
\caption{Node settings, visible when a node is selected}
\label{fig:anl-node-settings}
\end{figure}

\begin{figure}
\centering
\includegraphics[width=1\textwidth]{graphics/analysis-tools/visualisation-sidebar-peer-explorer.pdf}
\caption{Peer explorer}
\label{fig:anl-sim-peer-explorer}
\end{figure}

\begin{sidewaysfigure}
\centering
\vspace{15.2cm}
\includegraphics[width=1\textwidth]{graphics/analysis-tools/visualisation-stream-overlay.jpg}
\caption{Visualisation when a node starts simulated stream}
\label{fig:anl-sim-stream-active}
\end{sidewaysfigure}

\begin{figure}
\centering
\includegraphics[width=1\textwidth]{graphics/analysis-tools/visualisation-sidebar-channel-explorer.pdf}
\caption{Channel explorer}
\label{fig:anl-sim-channel-explorer}
\end{figure}

\begin{figure}
\centering
\includegraphics[width=1\textwidth]{graphics/analysis-tools/visualisation-sidebar-network-stats.pdf}
\caption{Network statistics}
\label{fig:anl-sim-network-stats}
\end{figure}

\begin{figure}
\centering
\includegraphics[width=1\textwidth]{graphics/analysis-tools/visualisation-sidebar-message-explorer.pdf}
\caption{Message explorer}
\label{fig:anl-sim-message-explorer}
\end{figure}

\begin{figure}
\centering
\includegraphics[width=1\textwidth]{graphics/analysis-tools/visualisation-sidebar-log-explorer.pdf}
\caption{Log entries of a node}
\label{fig:anl-sim-log-explorer}
\end{figure}


\clearpage

%\printindex

\printbibliography

\ifdeutsch
Alle URLs wurden zuletzt am 10.\,03.\,2019 geprüft.
\else
All links were last followed on March 10, 2019.
\fi

\pagestyle{empty}
\renewcommand*{\chapterpagestyle}{empty}
\Versicherung
\end{document}
