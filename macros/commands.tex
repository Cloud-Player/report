%wird fuer Tabellen benötigt (z.B. >{centering\RBS}p{2.5cm} erzeugt einen zentrierten 2,5cm breiten Absatz in einer Tabelle
\newcommand{\RBS}{\let\\=\tabularnewline}

%To avoid issues with Springer's \mathplus
%See also http://tex.stackexchange.com/q/212644/9075
\providecommand\mathplus{+}

%% typoraphisch richtige Abkürzungen
\newcommand{\zB}[0]{z.\,B.\xspace}
\newcommand{\bzw}[0]{bzw.\xspace}
\newcommand{\usw}[0]{usw.\xspace}
\renewcommand{\dh}[0]{d.\,h.\xspace}

%from hmks makros.tex - \indexify
\newcommand{\toindex}[1]{\index{#1}#1}
%
\newcommand{\dotcup}{\ensuremath{\,\mathaccent\cdot\cup\,}} %Tipp aus The Comprehensive LaTeX Symbol List
%
%Anstatt $|x|$ $\abs{x}$ verwenden. Die Betragsstriche skalieren automatisch, falls "x" etwas größer sein sollte...
\newcommand{\abs}[1]{\left\lvert#1\right\rvert}
%
%für Zitate
\newcommand{\citeS}[2]{\cite[S.~#1]{#2}}
\newcommand{\citeSf}[2]{\cite[S.~#1\,f.]{#2}}
\newcommand{\citeSff}[2]{\cite[S.~#1\,ff.]{#2}}
\newcommand{\vgl}{vgl.\ }
\newcommand{\Vgl}{Vgl.\ }
%
\newcommand{\commentchar}{\ensuremath{/\mkern-4mu/}}
\algrenewcommand{\algorithmiccomment}[1]{\hfill $\commentchar$ #1}

% Seitengrößen - Gegen Schusterjungen und Hurenkinder...
\newcommand{\largepage}{\enlargethispage{\baselineskip}}
\newcommand{\shortpage}{\enlargethispage{-\baselineskip}}

% CUSTOM
\newcommand{\itembf}[1]{\item\textbf{#1}\hspace{3mm}}
\newcommand*{\rom}[1]{\uppercase\expandafter{\romannumeral #1\relax}}

\newcommand{\alice}{\textit{Alice}\xspace}
\newcommand{\bob}{\textit{Bob}\xspace}
\newcommand{\claire}{\textit{Claire}\xspace}
\newcommand{\don}{\textit{Don}\xspace}
\newcommand{\eve}{\textit{Eve}\xspace}
\newcommand{\zoe}{\textit{Zoe}\xspace}
\newcommand{\signal}{\textit{Signal}\xspace}
\newcommand{\newbie}{\textit{Newbie}\xspace}
\newcommand{\router}{\textit{Router}\xspace}
\newcommand{\peer}{\textit{Peer}\xspace}

\newcommand{\maxConnections}{$C_{max}$\xspace}
\newcommand{\minConnectionGoal}{$C_{goal_{min}}$\xspace}
\newcommand{\maxConnectionGoal}{$C_{goal_{max}}$\xspace}
\newcommand{\outStreamConnections}{$S_{out}$\xspace}
\newcommand{\inStreamConnections}{$S_{in}$\xspace}
