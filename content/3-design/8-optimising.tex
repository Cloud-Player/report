\paragraph{Optimising}

Nodes in the mesh network do not maintain a fixed connection count, but rather work towards staying in a range between \minConnectionGoal and \maxConnectionGoal. There is also an upper bound \maxConnections, that none of the node's mechanics can exceed. If a node has less than \minConnectionGoal, it will have the \textit{desire} to acquire additional connections and if it has between \maxConnectionGoal and \maxConnections connections, it will want to disconnect its least performant neighbours. Candidates for connection acquisition are all second–degree nodes – or \text{2 hop neighbours} as the semantic equivalent from \gls{olsr} is called \ref{todo-olsr}. These are reported periodically by the \glspl{peer-update} as described in \ref{par:design-roles-peer-publish} along with an objective quality value. If one node was reported by multiple sources, its qualities are averaged. Further, nodes who are estimated to have no connection capacity will be punished and nodes who are reported to have the router role will be boosted. A ranking is built from these factors and a connection is negotiated with the best candidates. After the connection is successfully established, one of prior connections is disconnected to free up capacity on both sides.
Nodes will perform this procedure when they fall below their connection goal \minConnectionGoal (see \ref{todo-aquire-new-peers}) and in a coarser interval (see \ref{todo-try-other-peers}) to actively reposition themselves towards better neighbours.

For this continuous self optimisation to work, it is important to evaluate nodes as well as individual connections. Nodes measure the reliance of a delivery channel, the responsiveness of a node and the round–trip latency. Connections that do not transmit any data, expire after a certain time and second–degree nodes that expire after loosing all active connections.

- Convergence
