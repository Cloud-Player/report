\section{CLI}

\begin{figure}
\centering
\includegraphics[width=1\textwidth]{graphics/analysis-tools/cli.jpg}
\caption{CLI}
\label{fig:anl-cli}
\end{figure}

As the visualisation is doing some heavy lifting tasks to visualise the state of each node, it is not really suitable for large scenarios. Also using it for benchmarking with different parameters is a tedious task. It is very good for analysing and debugging small scenarios, yet for performance and scaling analysis it has its limits.

Thus, a \glsfirst{cli} has been developed which runs the simulation in a headless mode in a node.js environment. It accepts a path to a scenario JSON as paramter and also accepts a log level. By default the logging is turned off but can be turned on by the log level as parameter. The scenario is executed and the results are printed (\vref{fig:anl-cli})and saved as a comma separated values (csv) file. Also a JSON file is exported with the current state of the nodes. The file can be imported into the visualisation to analyse the state.

Not only is it faster to run a scenario, it also accepts a new type of configuration in the scenario which allows benchmarking. The parameter allows running one scenario in a batch job with different settings for Mitosis. The parameter can be configured in the scenario. When the benchmark is finished the CLI also saves the results as a csv and as a JSON file.