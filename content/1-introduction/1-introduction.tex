With the rise of social media and a wide adoption of connected devices that can create multimedia footage, the internet is shifting from a content consuming towards a content producing audience. Nowadays, everyone can take a photo or record a video and share it to an audience with the press of a button. 
Live broadcasting is also not reserved to professional media companies anymore. Network infrastructure is rapidly improving, especially the availability of mobile bandwidth. With the advent of 5G an estimated 2.7 billion people worldwide \cite{5g-stats} will have access to bandwidths of up to 10 Gigabits/s by 2025, which is \say{fast enough to download a 4K high-definition movie in 25 seconds}\cite{5g-promises}. In combination with ever more powerful consumer devices, users are able to produce and publish content in professional quality, e.g. the latest generation of smartphones can produce videos in resolutions up to 8k \cite{samsung-8k}.
Leveraging these new possibilities, platforms like Twitch, Periscope and IGTV have emerged that allow anyone with access to the internet to share live video with large audiences.

These platforms run vast cloud architectures to deliver video in real–time across the globe and already account for 58\% of global internet traffic \cite{phenomena-report}.
In 2017, users of the video streaming platform Periscope have streamed 350,000 hours of video per day \cite{periscope-stats} which amounts to around 70 Gigabits/s of continuous inbound traffic \footnote{Assuming an average bandwidth consumption of 5 Megabits/s per video.}.
Providing an infrastructure at that scale is very expensive and a non–trivial challenge.

Yet, a cloud architecture is not only expensive, it is also a single point of failure. If it is taken offline by failure or an attack, the service might not be usable anymore.
From the user standpoint, it also means losing control of data, because all the data is sent to the cloud, which is controlled by the provider. Thus, the provider has the power to monitor users and censor content based on political views or believes.

To mitigate the issues of a cloud architecture a lot of research has been put into Peer-to-Peer systems. Every peer can consume and provide content at the same time, thereby a central distribution point is not required anymore. Through the introduction of Napster in 1999, a Peer-to-Peer music sharing service, Peer-to-Peer systems have become popular. Since then, multiple other Peer-to-Peer systems like BitTorrent, Gnutella and most recently IPFS have emerged. BitTorrent is most known to be used for filesharing and supports more than a million daily active users \cite{bittorrent-stats}. Lesser known is the fact that the protocol is also used for server use–cases e.g. Facebook and Twitter are using it to update their backends \cite{bittorrent-facebook}.

In a recent attempt, BitTorrent has used their protocol for a Peer-to-Peer live video broadcasting service \cite{bittorrent-live}, but they shut down the service soon after \cite{bittorrent-live-shutdown}. However, BitTorrent has not been the first to use Peer-to-Peer technology for live video broadcasting. CoolStreaming, AnySee and Skype have also build live streaming solutions based on Peer-to-Peer technologies.

Through advances in browser technology, more specifically the support of WebRTC by all major browsers, it is possible to provide Peer-to-Peer live streaming as a web application. One major advantage of a web application is that the user does not have to install additional software. 
One service that is providing Peer-to-Peer live streaming in the browser is appear.in, which is a conferencing web app for smaller groups.

Given the recent advances in network bandwidth, performance of end user devices, browser support of WebRTC and the proof that Peer-to-Peer systems can scale up, the idea for a distributed media streaming application in the browser arose.

We propose a novel approach for distributed media streaming in the browser by using WebRTC technology for a multicast live stream scenario with a public audience.

\section{Motivation}
IPFS aims to \say{replace HTTP} and to provide a \say{peer-to-peer hypermedia protocol to make the web faster, safer, and more open} \cite{ipfs-website}.
Because of the bold statement of IPFS, our first approach was using IPFS to share small video snippets as a proof of concept. The sharing of video snippets has worked out quite well. Thus, our next step was using IPFS to share real-time videos. By using the HTML5 video recorder, a live video stream provided by the camera of the device, was chunked into small pieces and published via the \gls{pubsub} mechanism that is provided by IPFS. The \gls{pubsub} mechanism also notifies a peer when a new video chunk is available. As soon as the chunk is received by the peer, it is appended to the video element and played. However, to chunk the video into pieces the Media Source Extensions API has been used, which is not fully supported on Apple's iOS. Also, issues with video codecs and general performance and latency have not been satisfying.

This lead us to another approach, where we investigated how to realise live streaming with WebRTC. WebRTC allows one–to–many live streaming. Through restrictions of the peer, \say{many} is limited to a small amount. Thus, we put forward the hypothesis that we could use WebRTC to build a live streaming application, where we specify that each peer can receive one live stream and forward it to two other peers. By specifying, that a peer can forward a stream to two other peers, a binary tree can be created. This way, the stream can theoretically be forwarded to an infinite amount of peers.

To validate our hypothesis, we set up an experiment where we connected multiple peers using WebRTC. The peers were manually instructed to accept one stream and forward it to two other peers. 

The result of the experiment has been very promising and has proven our hypothesis. This motivated us to build a distributed media streaming application using WebRTC. 

\section{Outline}
The outline of the thesis is structured into five main chapters:
\begin{itemize}
    \itembf{Background} We start by explaining network fundamentals, including different network topologies and a selection of existing message routing protocols is examined. Next, \gls{p2p} systems are introduced like Napster, Gnutella and IPFS, several challenges of \gls{p2p} networks are examined and existing \gls{dht} algorithms like Chord and Kademlia are explained. Afterwards, browser capabilities that are required for the scope of the thesis are introduced, which include the Browser Media API and WebRTC. The chapter finishes with an analysis of existing live streaming solutions including centralised solutions like Twitch and Periscope but also decentralised approaches like CoolStreaming and AnyCast.
    
    \itembf{Design} This chapter proposes a design for a distributed media streaming application based on the findings of the background chapter. An overlay network is proposed, where each peer acts as an autonomous agent and tries to find best possible communication partners in the network without relying on global knowledge. Based on the overlay network a \gls{dag} is created, to propagate a video stream among the peers.
    \itembf{Implementation} In this chapter, the implementation of the design is explained and an insight of the inner workings is given. The implementation abstracts use–cases of a distributed media streaming application by providing an \gls{sdk}. Also, a sample application is introduced which has been built using the \gls{sdk}.
    \itembf{Proceedings} To analyse the implementation and for debugging during the development, several tools have been implemented. This chapter introduces the simulation, the visualisation and the \gls{cli} which has been used for benchmarking.
    \itembf{Analysis} The insights of the analysis-tools have been used to analyse the implementation. Mini and macros scenarios are examined to see how the implementation behaves under different circumstances. 
\end{itemize}
Finally, the thesis proposes future work and concludes with a résumé.
