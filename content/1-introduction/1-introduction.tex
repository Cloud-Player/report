\section{Introduction}

A new kind of decentralised consumer applications has emerged out of the problems cloud services are facing today: network congestion,
server costs and privacy concerns. \textit{dApps} address those issues by replacing centralised server architectures with a decentralised, self-organising network of clients. The three core areas of cloud computing, communication, storage and computation, are mapped into the decentralised paradigm. Clients are connected via peer-to-peer networking and do distributed content-addressed routing. Storage of media and application code is provided by decentralised file systems. Computation of business logic and sensitive transactions are spread across multiple clients that build a chain of trust. Technologies like IPFS \footnote{IPFS} (distributed file system), Chord \cite{chord} (distributed routing) and Ethereum \footnote{Ethereum} (distributed computation) enable new ways to build the web application of the future – a web application that implements privacy by design and has no single point of failure.

Consumer clients have become more capable thanks to advances in mobile hardware and the maturity of browser engines and their integration with OS-native functionality. This allows developers access to device sensors, GPU programming and opens up more possibilities with the network stack. Modern browsers expose interfaces for real-time communication between server and client, called WebSockets \cite{rfc6455}, and between clients, called WebRTC \cite{webrtc-w3c}. The latter enables connections to other clients without routing the data packages through a central server. Clients can therefore establish a connection to another client by only using the available network infrastructure.

The proposed thesis aims to apply the decentralised paradigm to an existing music player web application \footnote{cloudPlayer} that currently can play music from external providers like YouTube. The music player is to be extended by a peer-to-peer radio broadcasting feature. Any peer can act as a \textit{broadcaster} and control the media queue and playback of its \textit{receiving} peers. In addition, the \textit{broadcaster} can broadcast live audio content for vocal announcements and creative input. The radio feature will rely on decentralised networking to broadcast the content.

Two main challenges arise from this endeavour – optimising the peer-to-peer network for multi-cast streaming and securing the integrity and privacy of broadcasting streams.

\section{Random Intro quotes}
\say{It can be expected that the number and importance of such distributed applications will grow with time as the Internet is penetrating more and more into our lives.} (https://www.cs.vu.nl/pub/steen/papers/wip-newscast.pdf \S1)

\say{The popularity of peer-to-peer systems in the last couple of years illustrates how the Internet is gradually shifting toward a distributed system that supports more than only client-server applications.}(https://link.springer.com/content/pdf/10.1007\%2F978-3-540-25840-7\_6.pdf \S1)

\chapter{Motivation}
- Kostenfaktor
- Zensur bla
- Network Congestion / Alle leute machen Videostreaming. Mit multicast kannste dir so einige petabytes sparen. Pakete müssen nicht über nicht immer über die ganze welt hüpfen, bleibt in der regio woe es produziert/konsumiert wird. (Facebook paper referenzieren, Youtube/Netflix produzieren peta byte traffic)
