\subsection{Address}\label{sec:mit-address}
Addresses have an essential role for the communication between peers. Each peer in the network is addressable by its address in order to exchange data. 
However, the \gls{ip} address of a peer can not be used because a browser application can not open a direct connection to another peer by its \gls{ip}, as described in \vref{sec:webrtc}. 
Thus an address in Mitosis is a virtual abstraction of the physical address. 

The chosen address format is inspired by \textit{Multiaddr}\footnote{Multiaddr. URL: {https://multiformats.io/multiaddr}} which \say{is a format for encoding addresses from various well-established network protocols} \citet{multiaddr}. It is designed as a future proof and self–describable address that supports multiple transport protocols. 

\vref{lst:mit-address} shows the format of the Mitosis address and gives a few samples for valid addresses.

The first segment identifies the address protocol: \textit{Mitosis}. The following parameter identifies the \textit{version number}, which ensures that future protocol changes are possible and backwards compatibility can be offered.

The next parameter is the \textit{peer id}, that is generated by the peer itself when entering the network. A global unique identifier shall be chosen for the id to prevent name collisions. In the sample below, only simple ids have been chosen for demonstration purposes. 

The next segment defines the lower level protocol, but is optional. If it is set, a peer will use the according protocol for communication, else the peer chooses the best available connection.

Last but not least, an optional location can be set. The location represents the identifier of the connection. When a new connection is created, a unique connection id is assigned. As a peer can have multiple connections of the same protocol, the location can be used to address a specific connection. When the location is not set, the peer is free to choose which connection of the given protocol to use.

\vref{lst:mit-address} (\cref{line:mit-address-sample-start}-\cref{line:mit-address-sample-end}) gives examples for valid addresses. \cref{line:mit-address-sample-start}-\cref{line:mit-address-peer123} show address for a peer with the id \textit{p123}. \cref{line:mit-address-sample-start} will choose any existing connection to reach the peer, \cref{line:mit-address-webrtc} will choose any existing connection for the protocol \textit{webrtc-data} and \cref{line:mit-address-webrtc-id-1}-\cref{line:mit-address-peer123} require a specific \textit{webrtc-data} connection with the id \textit{c1234} or \textit{c5678}.

\paragraph{Supported protocols}
The protocols that are supported by Mitosis version 1 are:
\begin{itemize}
    \item ws / wss
    \item webrtc-data / webrtc-stream
    \item via / via-multi
\end{itemize}
Connection implementations, that are mapped to these protocols, are explained in \vref{sec:mit-connections}.
New protocols can be added simply by extending the Mitosis-Core \lstinline|ProtocolConnectionMap|. As soon as they are added, they can be used in the address as a protocol.

\begin{Listing}
\begin{lstlisting}[basicstyle=\footnotesize\ttfamily,xleftmargin=3em]
    /mitosis/{version}/{peer-id}/{protocol}?/{location}? |\label{line:mit-address-format}|
    
    /mitosis/1/p123 |\label{line:mit-address-sample-start}|
    /mitosis/1/p123/webrtc-data |\label{line:mit-address-webrtc}|
    /mitosis/1/p123/webrtc-data/c1234 |\label{line:mit-address-webrtc-id-1}|
    /mitosis/1/p123/webrtc-data/c5678 |\label{line:mit-address-peer123}|

    /mitosis/1/p007/wss/signal.mitosis.dev |\label{line:mit-address-sample-end}|
\end{lstlisting}
\label{lst:mit-address}
\caption{Mitosis address format and sample addresses}
\end{Listing}
