\subsection{Address}\label{sec:mit-address}
Addresses have an essential role for the communication between peers. Each peer in the network is addressable by its address in order to exchange data. 
However, the \gls{ip} address of a peer can not be used because a browser application can not open a direct connection to another peer by its \gls{ip}, as described in \vref{sec:webrtc}. 
Thus an address in Mitosis is a virtual abstraction of the physical address. 

The chosen address format is inspired by \textit{Multiaddr}\footnote{Multiaddr. URL: {https://multiformats.io/multiaddr}} which \say{is a format for encoding addresses from various well-established network protocols}\cite{multiaddr}. It was originally invented by Juan Benet (the founder of IPFS) with the goal to create future proof and self describable addresses that support multiple transport protocols. 

\vref{lst:mit-address} shows the format of the Mitosis address and gives a few samples for valid addresses.

The first parameter is the \textit{version number} of the Mitosis version that a peer is running. This ensures that future protocol changes are possible and backwards compatibility is theoretically possible.

After the version number comes the \textit{peer id}. The peer id is chosen by the peer itself when entering the network. A global unique identifier shall be chosen for the id to prevent name collisions. In the sample below only simple ids have been chosen for demonstration purposes. 

Next comes the protocol. A protocol is represented by a connection. Thus, when the protocol is set a peer will use the according connection for communication.
Since a peer shall support multiple connections the protocol parameter is optional. When the protocol is not set the peer takes any available connection that works best for him.

Last but not least an optional location can be set. The location represents the identifier of the connection. When a new connection is created, a peer unique connection id is set. As a peer can have multiple connections of the same protocol the location id can be used to address a specific connection. When the location is not set the peer decides by himself which connection to choose.

\vref{lst:mit-address} (\cref{line:mit-address-sample-start}-\cref{line:mit-address-sample-end}) gives examples for valid addresses. \cref{line:mit-address-sample-start}-\cref{line:mit-address-peer123} show address for a peer with the id \textit{p123}. \cref{line:mit-address-sample-start} will choose any existing connection to reach the peer, \cref{line:mit-address-webrtc} will choose any existing connection for the protocol \textit{webrtc-data} and \cref{line:mit-address-webrtc-id-1}-\cref{line:mit-address-peer123} require a specific \textit{webrtc-data} connection with the id \textit{c1234} or \textit{c5678}.

\paragraph{Supported protocols}
The protocols that are supported by Mitosis Version 1 are:
\begin{itemize}
    \item ws / wss
    \item webrtc-data / webrtc-stream
    \item via / via-multi
\end{itemize}
Connections that are mapped by these protocols are explained in \vref{sec:mit-connections}.
New protocols can be simply added by extending the Mitosis-Core \lstinline|ProtocolConnectionMap|. As soon as they are added they can be used in the address as protocol, otherwise when a protocol is not in the \lstinline|ProtocolConnectionMap| an error will be thrown.

\begin{Listing}
\begin{lstlisting}[basicstyle=\footnotesize\ttfamily,xleftmargin=3em]
    /mitosis/{version}/{peer-id}/{protocol}?/{location}? |\label{line:mit-address-format}|
    
    /mitosis/1/p123 |\label{line:mit-address-sample-start}|
    /mitosis/1/p123/webrtc-data |\label{line:mit-address-webrtc}|
    /mitosis/1/p123/webrtc-data/c1234 |\label{line:mit-address-webrtc-id-1}|
    /mitosis/1/p123/webrtc-data/c5678 |\label{line:mit-address-peer123}|

    /mitosis/1/p007/wss/signal.mitosis.dev |\label{line:mit-address-sample-end}|
\end{lstlisting}
\label{lst:mit-address}
\caption{Mitosis address format and sample addresses}
\end{Listing}
