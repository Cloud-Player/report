\subsection{Clock}\label{sec:mit-clock}
At the heart of Mitosis-Core is a custom clock. It supplies a heartbeat to the entire system. When the clock is stopped the inner workings of the algorithms are paused until the clock starts running again. However, stopping the clock does not mean the application execution is stopped. Only components that are subscribed to the clock's tick events are paused (e.g. the RoleManager as described in \vref{sec:mit-roleManager}). If a node receives a message and the clock is stopped, the message will still be processed.

In a real–world application, the clock would not need to be stopped, but for debugging purposes, stopping the clock is really handy — especially in the simulation environment. Another speciality is, that Mitosis-Core accepts an external clock during instantiation. In the simulation environment, this allows to have one global clock that all Mitosis instances are listing on. 
As a result, freezing the simulation is as simple as stopping the global clock. A developer then might check the state of all instances before they can start the clock again to continue the simulation.

The clock also abstracts the time unit of the machine and creates its own. The new time unit is generically called a tick $\ t $. By setting the interval of the clock, the tick interval can be configured. The default interval is $\ 1t/s $.

Clocks can be forked to create a new time unit relative to the interval of the master clock. For example when the master clock is running with $\ 1t/s $ the tick interval of the forked clock can be set to a $\ 0.1t/s $. Thus, a tick is triggered every $\ 10s $. However, a forked clock can never tick faster then the master clock.
