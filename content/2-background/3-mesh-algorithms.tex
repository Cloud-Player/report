\section{Mesh Network}
In a centralised environment each client communicates only with the central client. Therefore clients do not have to know other clients.
Clients in a decentralised network do not have a central counterpart. Thus a client has to discover other client and establish connections to them.

A more than 15 year old standard to achieve this is \gls{ip} multicast. The multicast concept operates on OSI-Layer 3 and is based on groups. Anyone can create a group with a unique address where others can receive packets from and send packets to. The use cases for multicast are unlimited like group chats, conference calls or tv broadcasting.
In contrast to unicast where a location is addressed a router has to duplicate packets and broadcast them. 
Due to technical limitations, like scaling issues and limited address space, it did not prove to be successful \cite{multicast}.

As multicast is not widely deplyoed a decentralised network has to be constructed as a overlay network on OSI-Layer 7—the application layer.

- mesh vs tree ist gut erklärt im AnySee paper / II Related Work