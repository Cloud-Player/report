\subsection{IPFS}\label{chap:IPFS}
\glsreset{ipfs}

The blockchain movement has sparked new interest in decentralised \gls{p2p} software architectures and networks in academia and the developer community \cite{medium-dnets}. Projects like MaidSafe\footnote{MaidSafe SAFE Network https://maidsafe.net}, Dat\footnote{Dat \gls{p2p} Protocol https://datproject.org} and IPFS\footnote{IPFS Protocol https://ipfs.io} are building protocols and \glspl{sdk} for applications and websites with decentralised storage and connectivity. One of these approaches is the self proclaimed \gls{ipfs}. \citet[\S.1]{ipfs-whitepaper} sees the current state of the web endangered by rising bandwidth demands, "disappearance of links" and a lack of upgradability. The proposed file system builds upon established distributed data exchange technologies and could pose an alternative to \gls{http}. It is composed of modular and exchangeable sub–protocols to ensure interoperability and future–proofing. Although \gls{ipfs} does not directly aim to facilitate media streaming, it suggests itself as a viable foundation for any distributed application. The advantage of \gls{ipfs} over similar projects is its clean separation of modules, which are implemented and maintained individually. Additionally, it has a \gls{js} implementation that works with any modern browser, instead of requiring extra software like the MaidSafe's custom browser.

\paragraph{Identity}
To ensure reliable identities and prevent impersonation, each node must be assigned a unique identifier and a private/public key pair. Similar to identity generation in S/Kademlia as described by \citet[\S4.1]{s_kademlia}, \gls{ipfs} requires nodes to solve a pair of crypto puzzles and use a hash of their public key as their identifier. This makes node creation computationally non-trivial and prevents adversaries from flooding the network with nodes \cite[\S3.2]{s_kademlia}.

\paragraph{Network}
\citet[\S3.2]{ipfs-whitepaper} proposes a network stack on top of flexible transport layers such as WebRTC \ref{webrtc} and uTP \cite{utp-micro-torrent-transport-protocol}. Transport reliability, message integrity and authenticity are all defined as optional attributes to be used if the use case requires it. Addresses in \gls{ipfs} are defined in the \lstinline|multiaddr| format which exposes multiple ways to reach a node, be it via proxies, \gls{nat} traversal or though different transport protocols.

\paragraph{Routing}
\gls{ipfs} defines a simple routing interface compliant to that of the Kademlia \gls{dht} \ref{TODO-kademlia}. Peers must be able to find other peers as well as retrieve values from said peers. Values are distinguish by size: up to 1kB is stored directly with peers providing a key and keys to larger values contain a set of seeding peers. \citet[\S3.5{ipfs-whitepaper} notes "different use cases will call for substantially different routing systems (e.g. DHT in wide network, static HT in local network)". To that end, the routing implementation should be exchangeable.

\paragraph{Data Distribution}
Files in \gls{ipfs} are defined made up of two abstraction layers. The lower layer consists of blocks of arbitrary binary data. \citet[\S3.4]{ipfs-whitepaper} introduces a BitTorrent inspired protocol called BitSwap. In BitSwap, each node tries to acquire blocks for itself and provide blocks to others. Nodes keep track of how much data others provide and request and punish leechers\footnote{The term leecher has been coined by the BitTorrent community and refers to users who take advantage of the network but do not contribute \cite[\S7.5]{tanenbaum_wetherall_2011}.}. Building upon the block layer, \gls{ipfs} uses a Merkle \gls{dag} to represent files and hierarchies \cite[\S3.5]{ipfs-whitepaper}. All content is addressed by its cryptographic hash \cite{content-centric-networking} and features like deduplication and versioning are built in.

\paragraph{Addressing}
But because files and links are addressed by their hash, modifications have a different hash – or put differently – objects are immutable. To enable persistent addresses for mutable objects, \citet[\S3.7]{ipfs-whitepaper} presents mutable links that can be published under a node's namespace. However, neither nodes nor pieces of content have humanly spellable identifiers and the author presents a workaround using \gls{dns} entries that point to \gls{ipfs} links.
