\subsection{IPFS}
\glsreset{ipfs}

The blockchain movement \cite{TODO} has sparked a new interest in decentralised \gls{p2p} network architectures \cite{medium-dnets}.

A more recent approach to \gls{p2p} data exchange is the self proclaimed \gls{ipfs}. \citet[\S.1]{ipfs-whitepaper} sees the current state of the web endangered by rising bandwidth demands, "disappearance of links" and a lack of upgradability. The proposed file system builds upon estabilshed distributed data exchange technologies and could pose an alternative to \gls{http}. It is composed of modular and exchangeable sub–protocols to ensure interoperability and future–proofing. Although \gls{ipfs} does not aim to facilitate media streaming, the interplay of its components suggests it as a viable foundation for any distributed application.
The individual components are described in the following sections.

\paragraph{Identity}
To ensure reliable identities and prevent impersonation, each node must be assigned a unique identifier and a private/public key pair. Similar to identity generation in S/Kademlia as described by \citet[\S4.1]{s_kademlia}, \gls{ipfs} requires nodes to solve a pair of crypto puzzles and use a hash of their public key as their identifier. This makes node creation computationally non-trivial and prevents adversaries from flooding the network with nodes \cite[\S3.2]{s_kademlia}.

\paragraph{Network}
\citet[\S3.2]{ipfs-whitepaper} proposes a network stack on top of flexible transport layers such as WebRTC \ref{webrtc} and uTP \cite{utp-micro-torrent-transport-protocol}. Transport reliability, message integrity and authenticity are all defined as optional attributes to be used if the use case requires it. Addresses in \gls{ipfs} are defined in the \lstinline|multiaddr| format which exposes multiple ways to reach a node, be it via proxies, \gls{nat} traversal or though different transport protocols.

\paragraph{Routing}
\gls{ipfs} defines a simple routing interface compliant to that of the Kademlia \gls{dht} \ref{TODO-kademlia}. Peers must be able to find other peers as well as retrieve values from said peers. Values are distinguish by size: up to 1kB is stored directly with peers providing a key and keys to larger values contain a set of seeding peers. \citet[\S3.5{ipfs-whitepaper} notes "different use cases will call for substantially different routing systems (e.g. DHT in wide network, static HT in local network)". To that end, the routing implementation should be exchangeable.

\paragraph{Data Distribution}
\gls{}

\paragraph{Merkle DAG}
