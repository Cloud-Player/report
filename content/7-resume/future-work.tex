\section{Future Work}
\begin{itemize}
    \item Persistence via IPFS
    \item End-To-End Encryption of the data-layer
    \item Crypto puzzle (proof of work) to enter system / send messages (spam prevention)
    \item Scaling the system with Kademlia
    \begin{itemize}
        \item Each node has a fingerprint id
        \item All routers together are forming the kademlia dht.
        \item A node id is saved in the dht as key. The value belonging to the key is the router to which it belongs 
        \item Each router can find any other node of the system by using the kademlia lookup. When Node A wants to find Node B it goes to its Router X. Router X is looking up the router node where Node B is stored as key. The lookup is returning the router where the node id is closest to the lookup-node-id-key in terms of the \gls{xor} distance. The value belonging to that key is the responsible Router Node Y for Node B. Node Y knows Node B as it is the cluster owner. Therefor the message from Node A to Node B is routed as following: \\ Node A -> -> Router X -> -> Router Y -> -> Node B
        \item Nodes are entering routers that are closest in terms of geographical location. However this is independent from the actual kademlia dht.
    \end{itemize}
\end{itemize}