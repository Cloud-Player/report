\section{Conclusion}
We started out with a simple experiment, manually connecting devices using \gls{webrtc} and instructing them to distribute a live video stream.
Our goal was to automate the process and build a system of self–organising agents, that could grow beyond the limits of a fully connected mesh.
We have looked into existing research and implementations of scalable mesh networks, common practices for distributed routing and approaches for live streaming in \gls{p2p} networks.

Our result is the design of a \gls{webrtc}–based overlay network with a custom routing protocol that allows distributing a live video feed by dynamically setting up a \glsfirst{dag} of stream connections. Nodes in the network act as independent agents and communicate with other nodes in their vicinity. Our implementation is inspired by the \glsfirst{bdi} model, as each node has the desire to find optimal connections and thereby strengthens the overall network.

We have verified our implementation by creating our own analysis tools. These helped in testing micro and macro scenarios, ingress of new peers and simulating network weak points. We also inspected the connection limits of our network and analysed the streaming topology. The simulation with its visualisation interface and benchmark \gls{cli}, has greatly helped in design, implementation and analysis of the entire system. Finally, it was possible to test how the implementation behaves under close to real–world conditions.

The results of our simulations have suggested that the system can serve one–to–many streaming scenarios for audiences with hundreds of nodes. A sample application (\vref{sec:symbiosis}) using the \textit{Mitosis} \gls{sdk} was implemented as a proof–of–concept. Preliminary tests with up to 25 real devices have shown stable video feeds under optimal network conditions. Real–world performance with greater variance in devices and higher user numbers has yet to be confirmed.

Both, simulations and real–world tests, have shown that modern browsers are a viable platform for application–layer multicast. It opens up new possibilities to web application developers, to integrate networking with multimedia \glspl{api} and application semantics. The distributed nature of our network, removes the need for server infrastructures, so independent and non–profit applications could integrate our software. For the user, \gls{mitosis}–based web apps are usable without the need for extra software and video stays in the user–space. By extending the implementation with further security measures, platforms can protect their users and content from the prying eyes of corporations and governments.

The \gls{mitosis} \gls{sdk} allows to build a wide range of application genres. Apart from the obvious video chat or broadcasting platform, gaming, e–learning or other genres are conceivable, especially if they leverage the possibilities of the data channel. We hope to inspire a new generation of distributed media live streaming platforms by contributing our open–source \gls{sdk}.
