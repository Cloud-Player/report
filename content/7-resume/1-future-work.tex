\section{Future Work}

\subsection{Persistence}
- ipfs

\subsection{Inter Cluster Communication}
- Scaling the system with Kademlia
    - Each node has a fingerprint id
    - All routers together are forming the kademlia dht.
    - A node id is saved in the dht as key. The value belonging to the key is the router to which it belongs 
    - Each router can find any other node of the system by using the kademlia lookup. When Node A wants to find Node B it goes to its Router X. Router X is looking up the router node where Node B is stored as key. The lookup is returning the router where the node id is closest to the lookup-node-id-key in terms of the \gls{xor} distance. The value belonging to that key is the responsible Router Node Y for Node B. Node Y knows Node B as it is the cluster owner. Therefor the message from Node A to Node B is routed as following: \\ Node A -> -> Router X -> -> Router Y -> -> Node B
    - Nodes are entering routers that are closest in terms of geographical location. However this is independent from the actual kademlia dht.

\subsection{Data Layer Encryption}
- End-To-End Encryption of the data-layer

\subsection{Identity}
- Crypto puzzle (proof of work) to enter system / send messages (spam prevention)

\subsection{Improved Signalling}
- geographical cluster selection
