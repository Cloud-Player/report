\section{Security Considerations}

\subsection{Encryption}
Whether end–to–end encrypted video streams are required, depends on the use case of the application. In a personal video chat application or the delivery of copyrighted material, it will be imperative to prevent unauthorised access to the material. However, in a publicly broadcasting application, where anyone is allowed to consume the media, end–to–end encryption will be less important.

In a \gls{p2p} delivery network, closed–audience use cases will contradict the core benefits. Why should other nodes contribute bandwidth to channels that are only viewable by closed audiences and not themselves? A more suitable application would be a \gls{p2p} multi–cast scenario, in which aiding nodes directly benefit if traffic is routed through them. It is possible to secure multi–casting with group–bound keys that are re–negotiated on node churn \cite{ip-multicast-sec}. However, that limits the size of the \gls{p2p} network to the size of the group, given that peers should directly benefit from the bandwidth they provide.

It can still be desirable to secure traffic between nodes, for instance, to disguise which channel is being consumed by which node. Yet, this would also require to obfuscate routing paths \cite[\S4]{tor-privacy} or add deceptive traffic \cite{swarm-screen}. Again, these measures are up to the application use case to justify.

On the transport layer, the \gls{webrtc} browser stack already provides mandatory media encryption using \gls{srtp} and a secure key exchange via \gls{dtls} \cref{par:webrtc-stack}. With this encryption, the transmission channel between two nodes is secured. However, since the Mitosis implementation actively pushes the streams to any interested node, the video channel is exposed by design and the encryption ineffective. If a channel where to be restricted to a closed audience, the video stream would need to be encrypted in the application layer and transported via \gls{sctp} instead. Although modern browsers provide a low–level cryptography \gls{api} this would certainly put extra strain on end–user devices. Additionally, a group key exchange would have to be secured – all in accordance with the golden rule of cryptography: \textit{Don't roll your own crypto} \cite{motherboard-dont-roll-your-own-crypto, schneider-dont-roll-your-own-crypto}.
With that in mind, it remains up to the application semantics to justify end–to–end encryption per use–case.

\subsection{Integrity}
Apart from restricting access with encryption, a video streaming application may need to protect against identity and content forgery. The current implementation of Mitosis guarantees neither. Looking at identity generation in S/Kademlia \vref{par:kademlia} and \gls{ipfs} \vref{chap:IPFS}, it has been explored to let nodes create their own identity with asymmetric keys and a self–certifying ids. Inflationary node generation is hindered by crypto puzzles and incentives for persistent identities. From an application perspective, it is complex problem to tie node identity to the identity of the human user in a distributed system. The blockchain and \gls{iam} community are pushing to implement decentralised identity systems using smart contracts \cite{eth-identity} and identity wallets \cite{gartner-iam}.
Once a system has established a trustworthy identity management, it can also provide content integrity guarantees. By letting a content producer digitally sign its work, others can verify that the content has not been tampered with. As \citet[\S7]{anatomy-personalized-livestreaming} demonstrate, an unsecured video transmission is easily intercepted and modified. Although this sort of attack is not directly harmful to viewers, it can still impact system performance and user experience.

\subsection{Vulnerability}
Distributed \gls{p2p} networks have the advantage of lacking a \textit{central point of failure}. This makes them more resilient against denial of service attacks by design. However, there are still common vulnerabilities in these systems, that have been observed in the wild and explored in related research.
\begin{itemize}
    \itembf{Eclipse Attack} \gls{p2p} networks where nodes can influence their own position (connections) can suffer from eclipse attacks \cite[\S3]{s_kademlia}. An adversary would create nodes and place them between the victims and the rest of the network. The victims can then be cut off from communications and would need to detect the attack themselves. In this implementation, nodes are continuously evaluating their neighbours and would report back to the signal if cut off from the network. Similarly, eclipsing nodes on the video streaming layer would push them towards seeking new connections. The eclipse attack would consequently work in the Mitosis network, yet only temporarily.
    \itembf{Sybil Attack} This attack tries to manipulate a reputation system in a \gls{p2p} network by creating enough artificial nodes. \citet[via BM08]{sybil-attack} shows it can be prevented by centralised certificate authorities or placing a computational price on creating new nodes, i.e. expecting proof of work. The Mitosis network is vulnerable to this attack, as node creation is unrestricted and the reporting of peer qualities is only validated by the connection quality of the reporting node. Nodes could easily be tricked into abandoning a connection if enough of its neighbours were malicious.
    \itembf{DDoS} Distributed denial of service attacks apply to \gls{p2p} as well as any other computer network \cite[\S2]{p2p-vulnerabilities}. As Mitosis relies on \gls{webrtc} connection negotiation to traverse \gls{nat} and firewall restrictions, \textit{out of protocol} attacks on nodes are unlikely. Within the protocol, nodes are certainly exposed to attackers flooding them with messages. The system however restricts the range of messages with \gls{ttl} settings. Further, steps to protect against this class of attacks would be to limit the rate at which messages are accepted or require the solution to a \textit{crypto puzzle} for each message.
\end{itemize}
