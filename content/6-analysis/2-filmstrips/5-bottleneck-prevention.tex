\subsection{Bottleneck prevention}

\begin{figure}[htb!]
  \centering
    \subfloat[]{\includegraphics[width=0.33\textwidth]{graphics/analysis/mini-scenarios/bottleneck-prevention/1.jpg} \label{fig:filmstrips-bottleneck-prevention-a}}
    \subfloat[]{\includegraphics[width=0.33\textwidth]{graphics/analysis/mini-scenarios/bottleneck-prevention/2.jpg} \label{fig:filmstrips-bottleneck-prevention-b}}
	\subfloat[]{\includegraphics[width=0.33\textwidth]{graphics/analysis/mini-scenarios/bottleneck-prevention/3.jpg} \label{fig:filmstrips-bottleneck-prevention-c}}
	\caption{Preventing a bottleneck}
\label{fig:filmstrips-bottleneck-prevention}
\end{figure}

An important requirement of a network is, that nodes can theoretically reach any other node in the network. This is guaranteed as long as a path exists. In fact, not only one path but multiple paths should exists to avoid critical nodes. A critical node is a node that others are dependent on to reach other parts of the network. \vref{fig:filmstrips-bottleneck-prevention-a} shows a scenario where node \textit{p5417} connects two networks. In case it crashes or leaves the network, the other peers would be disconnected from the main network.

To mitigate critical nodes, Mitosis is introducing a \textit{Router-Gravity} metric. The Router-Gravity metric is used during the peer selection process when a peer is acquiring new peers. A peer that is closer to the \router will deliver Router-Alive messages with a certain sequence number faster than peers that are further away from the \router.
During the acquisition process, a peer selects virtually connected peers with from its directly connected peers. The \textit{Via-Peers} of a peer who is delivering the Router-Alive message faster are ranked higher then those of slower peers or peers that are not delivering Router-Alive message at all.

\vref{fig:filmstrips-bottleneck-prevention-b} shows how the peers are connecting to the peers closer to the router. Thus they are creating new path to eliminate the critical node \textit{p5417} (\vref{fig:filmstrips-bottleneck-prevention-c}).
