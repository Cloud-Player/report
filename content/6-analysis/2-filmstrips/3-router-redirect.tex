\subsection{Join network II}

\begin{figure}[htb!]
  \centering
    \subfloat[]{\includegraphics[width=0.33\textwidth]{graphics/analysis/mini-scenarios/router-full-redirect/1.jpg} \label{fig:filmstrips-redirect-a}}
    \subfloat[]{\includegraphics[width=0.33\textwidth]{graphics/analysis/mini-scenarios/router-full-redirect/2.jpg} \label{fig:filmstrips-redirect-b}}
	\subfloat[]{\includegraphics[width=0.33\textwidth]{graphics/analysis/mini-scenarios/router-full-redirect/3.jpg} \label{fig:filmstrips-redirect-c}}
	\caption{Join network, router has no capacity}
\label{fig:filmstrips-redirect}
\end{figure}

A new peer, \textit{p6151}, is joining the network, therefore it tries to establish a connection to the \router peer \alice by sending her a connection offer \vref{fig:filmstrips-redirect-a}. 
Yet, \alice has already reached the maximum connection limit, that has been configured for this scenario $c_{max}=4$. Thus, \alice is rejecting the offer, but is simultaneously sending a selection of known peers as a \textit{Peer-Suggestion} message.

The new peer updates its PeerTable with the suggestions from \alice. To satisfy the connection goal, it sends connection offers to the suggested peer. In this case, it sends a connection offer to \claire. The offer is transmitted via \signal and \alice (\vref{fig:filmstrips-redirect-b}). 

\claire receives the offer and as she has one connection available she is sending back her answer to the new peer. The answer is transmitted via \alice and \signal. As soon as the new peer receives the answer, its establishes the connection (\vref{fig:filmstrips-redirect-c}).
