\subsection{Network split}

\begin{figure}[htb!]
  \centering
    \subfloat[]{\includegraphics[width=0.25\textwidth]{graphics/analysis/mini-scenarios/network-split/1.png} \label{fig:filmstrips-network-split-a}}
    \subfloat[]{\includegraphics[width=0.25\textwidth]{graphics/analysis/mini-scenarios/network-split/2.png} \label{fig:filmstrips-network-split-b}}
	\subfloat[]{\includegraphics[width=0.25\textwidth]{graphics/analysis/mini-scenarios/network-split/3.png} \label{fig:filmstrips-network-split-c}}
	\subfloat[]{\includegraphics[width=0.25\textwidth]{graphics/analysis/mini-scenarios/network-split/4.png} \label{fig:filmstrips-network-split-d}}
	\caption{Group of nodes looses contact to main network}
\label{fig:filmstrips-network-split}
\end{figure}

Mitosis tries to prevent network splits through the \textit{Router-Gravity} metric, yet it can not prevent that a group or a single peer eventually splits from the main network. 

\vref{fig:filmstrips-network-split-a} shows a scenario where a split was enforced by closing connections with the visualisation tool. As the nodes do not have a global knowledge of the network they do not know that they have lost the connection to the main network immediately.

Each \lstinline|RemotePeer| in the \lstinline|PeerTable| of a node has an expiration date based on its last seen. Last seen of of a RemotePeer is updated everytime a message has been received. When last seen exceeds a hard defined threshold, the peer is removed from the PeerTable.

The peer with the role \router, \alice, is constantly sending \routerAlive messages. Thus, the last seen of \alice is always be updated when any path from \alice to the peer exist. 
In \vref{fig:filmstrips-network-split-b} the last seen has exceeded the threshold and therefore the router peer \alice is removed from the PeerTable. Mitosis specifies that when a peer does not have any RemotePeer in its PeerTable with the role \router it is degraded to the role \newbie.

A peer with the role \peer is contacting the \signal peer to re-enter the network (\vref{fig:filmstrips-network-split-c}). As \alice has available connections, the some negotiation offers are accepted, thus a connection is established. The \routerAlive message is propagated in the network that has re-entered the network, thus all peers know that they have been absorbed into the main network (\vref{fig:filmstrips-network-split-d}).